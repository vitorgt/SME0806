\PassOptionsToPackage{unicode=true}{hyperref} % options for packages loaded elsewhere
\PassOptionsToPackage{hyphens}{url}
%
\documentclass[11pt,]{article}
\usepackage{lmodern}
\usepackage{amssymb,amsmath}
\usepackage{ifxetex,ifluatex}
\usepackage{fixltx2e} % provides \textsubscript
\ifnum 0\ifxetex 1\fi\ifluatex 1\fi=0 % if pdftex
  \usepackage[T1]{fontenc}
  \usepackage[utf8]{inputenc}
  \usepackage{textcomp} % provides euro and other symbols
\else % if luatex or xelatex
  \usepackage{unicode-math}
  \defaultfontfeatures{Ligatures=TeX,Scale=MatchLowercase}
\fi
% use upquote if available, for straight quotes in verbatim environments
\IfFileExists{upquote.sty}{\usepackage{upquote}}{}
% use microtype if available
\IfFileExists{microtype.sty}{%
\usepackage[]{microtype}
\UseMicrotypeSet[protrusion]{basicmath} % disable protrusion for tt fonts
}{}
\IfFileExists{parskip.sty}{%
\usepackage{parskip}
}{% else
\setlength{\parindent}{0pt}
\setlength{\parskip}{6pt plus 2pt minus 1pt}
}
\usepackage{hyperref}
\hypersetup{
            pdftitle={Trabalho 1 - SME0806 - Estatística Computacional},
            pdfauthor={Diego G. de Paulo (10857040); Caio Henrique M. Schiavo (11810602); Vitor Gratiere Torres (10284952); Douglas Sudré Souza (10733820); Bruno H.da S Justino (11031621)},
            pdfborder={0 0 0},
            breaklinks=true}
\urlstyle{same}  % don't use monospace font for urls
\usepackage[margin=1in]{geometry}
\usepackage{color}
\usepackage{fancyvrb}
\newcommand{\VerbBar}{|}
\newcommand{\VERB}{\Verb[commandchars=\\\{\}]}
\DefineVerbatimEnvironment{Highlighting}{Verbatim}{commandchars=\\\{\}}
% Add ',fontsize=\small' for more characters per line
\usepackage{framed}
\definecolor{shadecolor}{RGB}{248,248,248}
\newenvironment{Shaded}{\begin{snugshade}}{\end{snugshade}}
\newcommand{\AlertTok}[1]{\textcolor[rgb]{0.94,0.16,0.16}{#1}}
\newcommand{\AnnotationTok}[1]{\textcolor[rgb]{0.56,0.35,0.01}{\textbf{\textit{#1}}}}
\newcommand{\AttributeTok}[1]{\textcolor[rgb]{0.77,0.63,0.00}{#1}}
\newcommand{\BaseNTok}[1]{\textcolor[rgb]{0.00,0.00,0.81}{#1}}
\newcommand{\BuiltInTok}[1]{#1}
\newcommand{\CharTok}[1]{\textcolor[rgb]{0.31,0.60,0.02}{#1}}
\newcommand{\CommentTok}[1]{\textcolor[rgb]{0.56,0.35,0.01}{\textit{#1}}}
\newcommand{\CommentVarTok}[1]{\textcolor[rgb]{0.56,0.35,0.01}{\textbf{\textit{#1}}}}
\newcommand{\ConstantTok}[1]{\textcolor[rgb]{0.00,0.00,0.00}{#1}}
\newcommand{\ControlFlowTok}[1]{\textcolor[rgb]{0.13,0.29,0.53}{\textbf{#1}}}
\newcommand{\DataTypeTok}[1]{\textcolor[rgb]{0.13,0.29,0.53}{#1}}
\newcommand{\DecValTok}[1]{\textcolor[rgb]{0.00,0.00,0.81}{#1}}
\newcommand{\DocumentationTok}[1]{\textcolor[rgb]{0.56,0.35,0.01}{\textbf{\textit{#1}}}}
\newcommand{\ErrorTok}[1]{\textcolor[rgb]{0.64,0.00,0.00}{\textbf{#1}}}
\newcommand{\ExtensionTok}[1]{#1}
\newcommand{\FloatTok}[1]{\textcolor[rgb]{0.00,0.00,0.81}{#1}}
\newcommand{\FunctionTok}[1]{\textcolor[rgb]{0.00,0.00,0.00}{#1}}
\newcommand{\ImportTok}[1]{#1}
\newcommand{\InformationTok}[1]{\textcolor[rgb]{0.56,0.35,0.01}{\textbf{\textit{#1}}}}
\newcommand{\KeywordTok}[1]{\textcolor[rgb]{0.13,0.29,0.53}{\textbf{#1}}}
\newcommand{\NormalTok}[1]{#1}
\newcommand{\OperatorTok}[1]{\textcolor[rgb]{0.81,0.36,0.00}{\textbf{#1}}}
\newcommand{\OtherTok}[1]{\textcolor[rgb]{0.56,0.35,0.01}{#1}}
\newcommand{\PreprocessorTok}[1]{\textcolor[rgb]{0.56,0.35,0.01}{\textit{#1}}}
\newcommand{\RegionMarkerTok}[1]{#1}
\newcommand{\SpecialCharTok}[1]{\textcolor[rgb]{0.00,0.00,0.00}{#1}}
\newcommand{\SpecialStringTok}[1]{\textcolor[rgb]{0.31,0.60,0.02}{#1}}
\newcommand{\StringTok}[1]{\textcolor[rgb]{0.31,0.60,0.02}{#1}}
\newcommand{\VariableTok}[1]{\textcolor[rgb]{0.00,0.00,0.00}{#1}}
\newcommand{\VerbatimStringTok}[1]{\textcolor[rgb]{0.31,0.60,0.02}{#1}}
\newcommand{\WarningTok}[1]{\textcolor[rgb]{0.56,0.35,0.01}{\textbf{\textit{#1}}}}
\usepackage{graphicx,grffile}
\makeatletter
\def\maxwidth{\ifdim\Gin@nat@width>\linewidth\linewidth\else\Gin@nat@width\fi}
\def\maxheight{\ifdim\Gin@nat@height>\textheight\textheight\else\Gin@nat@height\fi}
\makeatother
% Scale images if necessary, so that they will not overflow the page
% margins by default, and it is still possible to overwrite the defaults
% using explicit options in \includegraphics[width, height, ...]{}
\setkeys{Gin}{width=\maxwidth,height=\maxheight,keepaspectratio}
\setlength{\emergencystretch}{3em}  % prevent overfull lines
\providecommand{\tightlist}{%
  \setlength{\itemsep}{0pt}\setlength{\parskip}{0pt}}
\setcounter{secnumdepth}{0}
% Redefines (sub)paragraphs to behave more like sections
\ifx\paragraph\undefined\else
\let\oldparagraph\paragraph
\renewcommand{\paragraph}[1]{\oldparagraph{#1}\mbox{}}
\fi
\ifx\subparagraph\undefined\else
\let\oldsubparagraph\subparagraph
\renewcommand{\subparagraph}[1]{\oldsubparagraph{#1}\mbox{}}
\fi

% set default figure placement to htbp
\makeatletter
\def\fps@figure{htbp}
\makeatother

\usepackage{enumerate}
\usepackage{hyperref}
\usepackage{booktabs}
\renewcommand{\and}{\\}
\usepackage{etoolbox}
\makeatletter
\providecommand{\subtitle}[1]{% add subtitle to \maketitle
  \apptocmd{\@title}{\par {\large #1 \par}}{}{}
}
\makeatother

\title{Trabalho 1 - SME0806 - Estatística Computacional}
\providecommand{\subtitle}[1]{}
\subtitle{Universidade de São Paulo}
\author{Diego G. de Paulo (10857040) \and Caio Henrique M. Schiavo (11810602) \and Vitor Gratiere Torres (10284952) \and Douglas Sudré Souza (10733820) \and Bruno H.da S Justino (11031621)}
\date{21/05/2021}

\begin{document}
\maketitle

{
\setcounter{tocdepth}{3}
\tableofcontents
}
\newpage

\hypertarget{introduuxe7uxe3o}{%
\section{Introdução}\label{introduuxe7uxe3o}}

Após a apresentação e explicação do conteúdo da disciplina, foi
solicitado, pelo professor, um trabalho com base em algoritmos de
amostras pseudo-aleatórias (Método da Transformação e Método da
Rejeição) e a aplicação do Método de Monte Carlo para estimar, por
simulações, parâmetros e funções de variáveis aleatórias.

A simulação de um modelo probabilístico consiste na geração de
mecanismos estocásticos e, em seguida, na observação do fluxo resultante
do modelo ao longo do tempo. No Método de Monte Carlo, nome originado
pelo uso da aleatoriedade e da natureza repetitiva das atividades
realizadas em cassinos em Monte Carlo, Mônaco, representa-se a solução
de um problema como um parâmetro de uma população hipotética e, que usa
uma sequência aleatória de números para construir uma amostra da
população da qual estimativas estatísticas desse parâmetro possam ser
obtidas. Halton (1970).

Neste trabalho, também nos deparamos com distribuições do tipo
Log-Normal, caracterizada pela propriedade que os logaritmos dos valores
seguem uma distribuição normal e pela forte assimetria positiva, dada
pela ocorrência de uma grande quantidade de valores baixos e uma pequena
quantidade de valores altos a muito altos. (Koch e Link, 1970, p.~213)

\newpage

\hypertarget{exercuxedcio-1}{%
\section{Exercício 1}\label{exercuxedcio-1}}

\hypertarget{motivauxe7uxe3o}{%
\subsection{Motivação}\label{motivauxe7uxe3o}}

~

O intuito deste exercício é gerar uma amostra pseudo-aleatória de
\(f(x)\) dada por \(f(x) \propto q(x) = e^{\frac{-|x|^3}{3}}\).

\hypertarget{metodologia}{%
\subsection{Metodologia}\label{metodologia}}

~

Para gerar tal amostra, foi selecionado o método da rejeição. Este
método é descrito por:

A seleção de uma variável aleatória \(Y\), com função de densidade dada
por \(g(y)\) amostrável. Além disso, há a suposição:

\begin{itemize}
\tightlist
\item
  \(\frac{f(x)}{g(x)} \leq M\), \(1 \leq M < \infty\)
\end{itemize}

E, por recomendação, toma-se
\(M = max_x\left( \frac{f(x)}{g(x)} \right)\)

\hypertarget{resoluuxe7uxe3o}{%
\subsection{Resolução}\label{resoluuxe7uxe3o}}

~

Para o exercício em questão, seleciona-se \(Y\), tal que
\(Y \sim Laplace(0,1)\) que tem a função de probabilidade dada por:
\(g(y) = \frac{1}{2} e^{-|y|}, y \in \mathbb{R}\). Para obter M, tem-se:
\[M = max_x\left( \frac{f(x)}{g(x)} \right) \iff \frac{d\left(\frac{e^{\frac{-|x|^3}{3}}}{\frac{1}{2} e^{-|x|}}  \right)}{dx} = 0\]
Calculando a derivada de \(\frac{f(x)}{g(x)}\):

\[
\frac{d\left(\frac{f(x)}{g(x)}
\right)}{dx} =
\frac{d\left(\frac{e^{\frac{-|x|^3}{3}}}{\frac{1}{2} e^{-|x|}}  \right)}{dx} =
\frac{2e^{\frac{-|x|^3+3|x|}{3}}x(-x^2+1)}{|x|}
\]

Igualando a zero:

\[\frac{2e^{\frac{-|x|^3+3|x|}{3}}x(-x^2+1)}{|x|} = 0\] Como solução
para esta equação tem-se: \(x = -1, 0, 1\), afim de não postergar o
cálculo e partir para o gerador de amostra pseudo-aleatória,
seleciona-se, dos pontos críticos, apenas os pontos de máximo em
\(x = -1, 1\). Sendo assim obtém-se:
\(M = max_x\left( \frac{f(x)}{g(x)} \right) = 2e^{\frac{2}{3}}\).
Finalizada a etapa de seleção das variáveis, segue a apliacação das
etapas:

\begin{itemize}
\tightlist
\item
  1º: Gerar uma amostra de \(Y\)
\item
  2º: Gerar \(u \sim U(0,1)\)
\item
  3º: Se \(u \leq \frac{f(y)}{Mg(y)}\) faça \(x = y\), Caso contrário
  retornar ao primeiro passo.
\item
  4º: Repita os passos anteriores até obter n observações necessárias.
\end{itemize}

\hypertarget{gruxe1ficos}{%
\subsection{Gráficos}\label{gruxe1ficos}}

~

Abaixo, para melhor visualização dos resultados obtidos matematicamente,
estão exibidos os gráficos das funções \(f(x)\), \(g(y)\), assim como os
gráficos de \(\frac{f(x)}{g(x)}\), para que os pontos critícos possam
ser observados, e o gráfico sobreposto de \(f(x)\) e \(Mg(x)\), para
notar o envolapamento de \(f(x)\) por \(Mg(x)\).

\begin{Shaded}
\begin{Highlighting}[]
\NormalTok{gx <-}\StringTok{ }\ControlFlowTok{function}\NormalTok{(x) \{}

  \KeywordTok{return}\NormalTok{(}\FloatTok{0.5} \OperatorTok{*}\StringTok{ }\KeywordTok{exp}\NormalTok{(}\OperatorTok{-}\KeywordTok{abs}\NormalTok{(x)))}

\NormalTok{\}}

\NormalTok{fx <-}\StringTok{ }\ControlFlowTok{function}\NormalTok{(x) \{}

  \KeywordTok{return}\NormalTok{(}\KeywordTok{exp}\NormalTok{(}\OperatorTok{-}\NormalTok{(}\KeywordTok{abs}\NormalTok{(x)}\OperatorTok{^}\DecValTok{3}\NormalTok{)}\OperatorTok{/}\DecValTok{3}\NormalTok{))}

\NormalTok{\}}
\end{Highlighting}
\end{Shaded}

\begin{Shaded}
\begin{Highlighting}[]
\NormalTok{fgx <-}\StringTok{ }\ControlFlowTok{function}\NormalTok{(x)\{}

  \KeywordTok{return}\NormalTok{(}\KeywordTok{fx}\NormalTok{(x)}\OperatorTok{/}\KeywordTok{gx}\NormalTok{(x))}

\NormalTok{\}}

\NormalTok{M <-}\StringTok{ }\KeywordTok{fgx}\NormalTok{(}\OperatorTok{-}\DecValTok{1}\NormalTok{)}



\NormalTok{M_gx <-}\StringTok{ }\ControlFlowTok{function}\NormalTok{(x, M) \{}
  \KeywordTok{return}\NormalTok{(M }\OperatorTok{*}\StringTok{ }\KeywordTok{gx}\NormalTok{(x))}
\NormalTok{\}}
\KeywordTok{par}\NormalTok{(}\DataTypeTok{mfrow=}\KeywordTok{c}\NormalTok{(}\DecValTok{2}\NormalTok{,}\DecValTok{2}\NormalTok{))}
\KeywordTok{curve}\NormalTok{(fx, }\DecValTok{-5}\NormalTok{, }\DecValTok{5}\NormalTok{, }\DataTypeTok{xlab =} \StringTok{"x"}\NormalTok{, }\DataTypeTok{ylab =} \StringTok{"f(x)"}\NormalTok{,}
      \DataTypeTok{col =} \StringTok{"darkgrey"}\NormalTok{, }\DataTypeTok{lwd =} \DecValTok{2}\NormalTok{, }\DataTypeTok{main =} \StringTok{"Gráfico de f(x)"}\NormalTok{)}

\KeywordTok{curve}\NormalTok{(gx, }\DecValTok{-5}\NormalTok{, }\DecValTok{5}\NormalTok{, }\DataTypeTok{xlab =} \StringTok{"y"}\NormalTok{, }\DataTypeTok{ylab =} \StringTok{"g(y)"}\NormalTok{,}
      \DataTypeTok{col =} \StringTok{"darkgrey"}\NormalTok{, }\DataTypeTok{lwd =} \DecValTok{2}\NormalTok{, }\DataTypeTok{main =} \StringTok{"Gráfico de g(y)"}\NormalTok{)}

\KeywordTok{curve}\NormalTok{(fgx, }\DecValTok{-4}\NormalTok{, }\DecValTok{4}\NormalTok{, }\DataTypeTok{ylab =} \StringTok{"f(x)/g(x)"}\NormalTok{, }\DataTypeTok{main =} \StringTok{"Gráfico de f(x)/g(x)"}\NormalTok{)}
\KeywordTok{points}\NormalTok{(}\KeywordTok{c}\NormalTok{(}\OperatorTok{-}\DecValTok{1}\NormalTok{, }\DecValTok{1}\NormalTok{), }\KeywordTok{c}\NormalTok{(M, M), }\DataTypeTok{pch =} \DecValTok{20}\NormalTok{, }\DataTypeTok{col =} \StringTok{"blue"}\NormalTok{)}
\KeywordTok{abline}\NormalTok{(}\DataTypeTok{h =}\NormalTok{ M, }\DataTypeTok{lty =} \DecValTok{2}\NormalTok{, }\DataTypeTok{col =} \StringTok{"blue"}\NormalTok{)}
\KeywordTok{segments}\NormalTok{(}\KeywordTok{c}\NormalTok{(}\OperatorTok{-}\DecValTok{1}\NormalTok{, }\DecValTok{1}\NormalTok{), }\KeywordTok{c}\NormalTok{(}\DecValTok{0}\NormalTok{, }\DecValTok{0}\NormalTok{), }\KeywordTok{c}\NormalTok{(}\OperatorTok{-}\DecValTok{1}\NormalTok{, }\DecValTok{1}\NormalTok{), }\KeywordTok{c}\NormalTok{(M, M), }\DataTypeTok{lty =} \DecValTok{2}\NormalTok{, }\DataTypeTok{col =} \StringTok{"blue"}\NormalTok{)}

\KeywordTok{curve}\NormalTok{(}\KeywordTok{M_gx}\NormalTok{(x, M), }\DecValTok{-5}\NormalTok{, }\DecValTok{5}\NormalTok{, }\DataTypeTok{col =} \StringTok{"red"}\NormalTok{, }\DataTypeTok{ylab =} \StringTok{"f(x) e M*g(x)"}\NormalTok{, }\DataTypeTok{main =} \StringTok{"Gráfico de f(x) e Mg(x)"}\NormalTok{)}
\KeywordTok{curve}\NormalTok{(}\KeywordTok{fx}\NormalTok{(x), }\DataTypeTok{add =} \OtherTok{TRUE}\NormalTok{)}
\KeywordTok{legend}\NormalTok{(}\StringTok{"topright"}\NormalTok{, }\KeywordTok{c}\NormalTok{(}\StringTok{"f(x)"}\NormalTok{, }\StringTok{"M*g(x)"}\NormalTok{), }\DataTypeTok{col =} \KeywordTok{c}\NormalTok{(}\StringTok{"black"}\NormalTok{, }\StringTok{"red"}\NormalTok{),}
\DataTypeTok{lty =} \DecValTok{1}\NormalTok{, }\DataTypeTok{bty =} \StringTok{"n"}\NormalTok{)}
\end{Highlighting}
\end{Shaded}

\includegraphics{trab1_files/figure-latex/unnamed-chunk-2-1.pdf}

\newpage

A seguir, é possível observar histogramas e boxplots para cada tamanho
de amostra \(n = (50, 100, 400)\) das amostras pseudo-aleatórias geradas
da \(f(x)\). É possível notar, observando o gráfico, que a medida em que
se aumenta o tamanho da amostra mais próximo da simetria observada na
função, dispõe-se a amostra.

\begin{Shaded}
\begin{Highlighting}[]
\NormalTok{gerador <-}\StringTok{ }\ControlFlowTok{function}\NormalTok{(n)\{}

\NormalTok{  nger <-}\StringTok{ }\NormalTok{n0 <-}\StringTok{ }\DecValTok{0}
\NormalTok{  ax <-}\StringTok{ }\KeywordTok{c}\NormalTok{()}
  \ControlFlowTok{while}\NormalTok{ (n0 }\OperatorTok{<}\StringTok{ }\NormalTok{n) \{}
\NormalTok{    rej <-}\StringTok{ }\OtherTok{TRUE}
    \ControlFlowTok{while}\NormalTok{(rej) \{}
\NormalTok{      nger <-}\StringTok{ }\NormalTok{nger }\OperatorTok{+}\StringTok{ }\DecValTok{1}
\NormalTok{      u <-}\StringTok{ }\KeywordTok{runif}\NormalTok{(}\DecValTok{1}\NormalTok{)}
\NormalTok{      y <-}\StringTok{ }\KeywordTok{ifelse}\NormalTok{(u }\OperatorTok{<=}\StringTok{ }\FloatTok{0.5}\NormalTok{, }\KeywordTok{log}\NormalTok{(}\DecValTok{2} \OperatorTok{*}\StringTok{ }\NormalTok{u), }\OperatorTok{-}\KeywordTok{log}\NormalTok{(}\DecValTok{2} \OperatorTok{*}\StringTok{ }\NormalTok{(}\DecValTok{1} \OperatorTok{-}\StringTok{ }\NormalTok{u)))}

      \ControlFlowTok{if}\NormalTok{ (M }\OperatorTok{*}\StringTok{ }\KeywordTok{runif}\NormalTok{(}\DecValTok{1}\NormalTok{) }\OperatorTok{<=}\StringTok{ }\KeywordTok{fgx}\NormalTok{(y)) \{}
\NormalTok{        n0 <-}\StringTok{ }\NormalTok{n0 }\OperatorTok{+}\StringTok{ }\DecValTok{1}
\NormalTok{        ax[n0] <-}\StringTok{ }\NormalTok{y}
\NormalTok{        rej <-}\StringTok{ }\OtherTok{FALSE}
\NormalTok{      \}}
\NormalTok{    \}}
\NormalTok{  \}}

\NormalTok{  result <-}\StringTok{ }\KeywordTok{list}\NormalTok{(}\DataTypeTok{sample =}\NormalTok{ ax, }\DataTypeTok{n_gerado =}\NormalTok{ nger)}
  \KeywordTok{return}\NormalTok{(result)}

\NormalTok{\}}

\NormalTok{amostra_gerada_}\DecValTok{50}\NormalTok{ <-}\StringTok{ }\KeywordTok{gerador}\NormalTok{(}\DecValTok{50}\NormalTok{)}
\NormalTok{ax_}\DecValTok{50}\NormalTok{ <-}\StringTok{ }\NormalTok{amostra_gerada_}\DecValTok{50}\OperatorTok{$}\NormalTok{sample}

\NormalTok{amostra_gerada_}\DecValTok{100}\NormalTok{ <-}\StringTok{ }\KeywordTok{gerador}\NormalTok{(}\DecValTok{100}\NormalTok{)}
\NormalTok{ax_}\DecValTok{100}\NormalTok{ <-}\StringTok{ }\NormalTok{amostra_gerada_}\DecValTok{100}\OperatorTok{$}\NormalTok{sample}

\NormalTok{amostra_gerada_}\DecValTok{400}\NormalTok{ <-}\StringTok{ }\KeywordTok{gerador}\NormalTok{(}\DecValTok{400}\NormalTok{)}
\NormalTok{ax_}\DecValTok{400}\NormalTok{ <-}\StringTok{ }\NormalTok{amostra_gerada_}\DecValTok{400}\OperatorTok{$}\NormalTok{sample}

\KeywordTok{par}\NormalTok{(}\DataTypeTok{mfrow=}\KeywordTok{c}\NormalTok{(}\DecValTok{2}\NormalTok{,}\DecValTok{2}\NormalTok{))}
\KeywordTok{hist}\NormalTok{(ax_}\DecValTok{50}\NormalTok{, }\DataTypeTok{freq =} \OtherTok{FALSE}\NormalTok{, }\DataTypeTok{main =} \StringTok{"Histograma da amostra n = 50"}\NormalTok{,}
     \DataTypeTok{xlab =} \StringTok{"x"}\NormalTok{, }\DataTypeTok{ylab =} \StringTok{"Densidade"}\NormalTok{,}
     \DataTypeTok{col=}\StringTok{"orange"}\NormalTok{, }\DataTypeTok{border=}\StringTok{"brown"}\NormalTok{)}
\KeywordTok{hist}\NormalTok{(ax_}\DecValTok{100}\NormalTok{, }\DataTypeTok{freq =} \OtherTok{FALSE}\NormalTok{, }\DataTypeTok{main =} \StringTok{"Histograma da amostra n = 100"}\NormalTok{,}
     \DataTypeTok{xlab =} \StringTok{"x"}\NormalTok{, }\DataTypeTok{ylab =} \StringTok{"Densidade"}\NormalTok{,}
     \DataTypeTok{col=}\StringTok{"orange"}\NormalTok{, }\DataTypeTok{border=}\StringTok{"brown"}\NormalTok{)}
\KeywordTok{hist}\NormalTok{(ax_}\DecValTok{400}\NormalTok{, }\DataTypeTok{freq =} \OtherTok{FALSE}\NormalTok{, }\DataTypeTok{main =} \StringTok{"Histograma da amostra n = 400"}\NormalTok{,}
     \DataTypeTok{xlab =} \StringTok{"x"}\NormalTok{, }\DataTypeTok{ylab =} \StringTok{"Densidade"}\NormalTok{,}
     \DataTypeTok{col=}\StringTok{"orange"}\NormalTok{, }\DataTypeTok{border=}\StringTok{"brown"}\NormalTok{)}

\NormalTok{amostra <-}\StringTok{ }\KeywordTok{c}\NormalTok{(ax_}\DecValTok{50}\NormalTok{, ax_}\DecValTok{100}\NormalTok{, ax_}\DecValTok{400}\NormalTok{)}
\NormalTok{tamanho <-}\StringTok{ }\KeywordTok{c}\NormalTok{(}\KeywordTok{rep}\NormalTok{(}\DecValTok{50}\NormalTok{, }\KeywordTok{length}\NormalTok{(ax_}\DecValTok{50}\NormalTok{)),}
             \KeywordTok{rep}\NormalTok{(}\DecValTok{100}\NormalTok{, }\KeywordTok{length}\NormalTok{(ax_}\DecValTok{100}\NormalTok{)),}
             \KeywordTok{rep}\NormalTok{(}\DecValTok{400}\NormalTok{, }\KeywordTok{length}\NormalTok{(ax_}\DecValTok{400}\NormalTok{)))}

\NormalTok{df <-}\StringTok{ }\KeywordTok{data.frame}\NormalTok{(amostra, tamanho)}

\KeywordTok{boxplot}\NormalTok{(amostra}\OperatorTok{~}\NormalTok{tamanho,}
\DataTypeTok{data=}\NormalTok{df,}
\DataTypeTok{main=}\StringTok{"Diferentes Boxplots para cada tamanho de amostra"}\NormalTok{,}
\DataTypeTok{xlab=}\StringTok{"Tamanho da amostra"}\NormalTok{,}
\DataTypeTok{ylab=}\StringTok{"Observações"}\NormalTok{,}
\DataTypeTok{col=}\StringTok{"orange"}\NormalTok{,}
\DataTypeTok{border=}\StringTok{"brown"}
\NormalTok{)}
\end{Highlighting}
\end{Shaded}

\includegraphics{trab1_files/figure-latex/unnamed-chunk-3-1.pdf}

\newpage

\hypertarget{exercuxedcio-2}{%
\section{Exercício 2}\label{exercuxedcio-2}}

\hypertarget{motivauxe7uxe3o-1}{%
\subsection{Motivação}\label{motivauxe7uxe3o-1}}

Uma aproximação pode ser obtida utilizando simulações de amostras de (X,
Y). Vamos gerar observações do par (X, Y) com as expressões que estão no
enunciado.

\hypertarget{resoluuxe7uxe3o-1}{%
\subsection{Resolução}\label{resoluuxe7uxe3o-1}}

\begin{Shaded}
\begin{Highlighting}[]
\NormalTok{R <-}\StringTok{ }\DecValTok{10000}
\NormalTok{S <-}\StringTok{ }\DecValTok{200}

\NormalTok{mean_hat <-}\StringTok{ }\KeywordTok{c}\NormalTok{()}
\ControlFlowTok{for}\NormalTok{(i }\ControlFlowTok{in} \DecValTok{1}\OperatorTok{:}\NormalTok{R)\{}
\NormalTok{  x <-}\StringTok{ }\KeywordTok{rlnorm}\NormalTok{(S,}\DecValTok{0}\NormalTok{,}\DecValTok{1}\NormalTok{)}
\NormalTok{  erro <-}\StringTok{ }\KeywordTok{rnorm}\NormalTok{(S,}\DecValTok{0}\NormalTok{,}\DecValTok{1}\NormalTok{)}
\NormalTok{  y <-}\StringTok{ }\KeywordTok{exp}\NormalTok{(}\DecValTok{9}\OperatorTok{+}\DecValTok{3}\OperatorTok{*}\KeywordTok{log}\NormalTok{(x)}\OperatorTok{+}\NormalTok{erro)}
\NormalTok{  mean_hat[i] <-}\StringTok{ }\KeywordTok{mean}\NormalTok{(y}\OperatorTok{/}\NormalTok{x)}
\NormalTok{\}}
\end{Highlighting}
\end{Shaded}

Estimação Pontual

Na estimação pontual desejamos encontrar um único valor numérico que
esteja bastante próximo do verdadeiro valor do parâmetro.

\begin{Shaded}
\begin{Highlighting}[]
\KeywordTok{mean}\NormalTok{(mean_hat)}
\end{Highlighting}
\end{Shaded}

\begin{verbatim}
## [1] 99854.96
\end{verbatim}

Estimação Intervalar

Embora os estimadores pontuais especifiquem um único valor para o
parâmetro, diferentes amostras levam a diferentes estimativas, pois o
estimador é uma função de uma amostra aleatória. E, estimar um parâmetro
através de um único valor não permite julgar a magnitude do erro que
podemos estar cometendo.

Daí, surge a ideia de contruir um intervalo de valores que tenha uma
alta probabilidade de conter o verdadeiro valor do parâmetro (denominado
intervalo de confiança).

\begin{Shaded}
\begin{Highlighting}[]
\KeywordTok{quantile}\NormalTok{(mean_hat, }\KeywordTok{c}\NormalTok{(}\FloatTok{0.025}\NormalTok{, }\FloatTok{0.975}\NormalTok{))}
\end{Highlighting}
\end{Shaded}

\begin{verbatim}
##      2.5%     97.5%
##  44054.94 255157.93
\end{verbatim}

Poderiamos resolver de diversas outras formas, de maneira que, com o
método utilizado, estamos 95\% confiantes de que o intervalo de 43703.07
a 252929.58 realmente contém o verdadeiro valor de p.~

\newpage

\hypertarget{exercuxedcio-3}{%
\section{Exercício 3}\label{exercuxedcio-3}}

\hypertarget{motivauxe7uxe3o-2}{%
\subsection{Motivação}\label{motivauxe7uxe3o-2}}

~

Neste presente exercício busca-se trabalhar com a distribuição poisson e
estudar os Erros do Tipo I e II para esta distribuição utilizando o
Método de Monte Carlo.

\hypertarget{resoluuxe7uxe3o-2}{%
\subsection{Resolução}\label{resoluuxe7uxe3o-2}}

Para facilitar a compreensão, será indicado alguns aspectos importantes
de tal distribuição. Dado \(X \sim Poisson(\lambda)\), \(X_1, ..., X_n\)
é uma amostra aleatória de \(X\) de tamanho \(n\).

\begin{itemize}
\item
  \(\mathbb{P}(X=k)=\frac{e^{-\lambda}\lambda^x}{x!}\), para
  \(\lambda > 0\) e \(x = 0, 1, 2, ...\);
\item
  \(\mathbb{E}(X) = \lambda\);
\item
  \(\text{Var}(X) = \lambda\);
\item
  \(\hat{\lambda}_{EMV} = \frac{1}{n}\sum_{i=1}^n X_i = \bar{X}\), em
  que \(k_i\) é a i-ésima observação amostral de \(X_i\);
\end{itemize}

Pela propriedade da eficiêcnia de um estimador de máxima
verossimilhança, tem-se o resultado:

\[\sqrt{n}(\hat{\lambda}_{EMV}-\lambda)\xrightarrow{_D}\mathcal{N}(0,\mathcal{IF}(\lambda)^{-1})\]
em que \(\mathcal{IF}(\lambda) = \frac{1}{\lambda}\)

Para o teste de hipótese:

\begin{itemize}
\tightlist
\item
  \(H_0:\) \(\lambda = 2\)
\item
  \(H_A:\) \(\lambda > 2\)
\end{itemize}

A Estatística do Teste \(T\) pode ser definida a partir do resultado da
efiência do estimador de máxima verossimilhança.

\[T = \sqrt{n}(\hat{\lambda}_{EMV}-2)\xrightarrow[\text{Sob } H_0]{D}\mathcal{N}\left(0,\frac{1}{2} \right)\]

\begin{Shaded}
\begin{Highlighting}[]
\NormalTok{R <-}\StringTok{ }\DecValTok{10000}
\NormalTok{lambda <-}\StringTok{ }\DecValTok{2}
\NormalTok{po_sample <-}\StringTok{ }\ControlFlowTok{function}\NormalTok{(R, lambda, n)\{}

\NormalTok{  pvalue <-}\StringTok{ }\KeywordTok{c}\NormalTok{()}
\NormalTok{  aceptance <-}\StringTok{ }\KeywordTok{c}\NormalTok{()}
\NormalTok{  medias_v <-}\StringTok{ }\KeywordTok{c}\NormalTok{()}
\NormalTok{  poder_dif <-}\StringTok{ }\KeywordTok{seq}\NormalTok{(}\FloatTok{2.2}\NormalTok{, }\DecValTok{4}\NormalTok{, }\DataTypeTok{length.out =} \DecValTok{5}\NormalTok{)}\OperatorTok{-}\DecValTok{2}
\NormalTok{  power_}\DecValTok{1}\NormalTok{ <-}\StringTok{ }\KeywordTok{c}\NormalTok{()}

  \ControlFlowTok{for}\NormalTok{ (i }\ControlFlowTok{in} \DecValTok{1}\OperatorTok{:}\NormalTok{R) \{}

\NormalTok{    sample_p <-}\StringTok{ }\KeywordTok{rpois}\NormalTok{(n, lambda)}
\NormalTok{    media <-}\StringTok{ }\KeywordTok{mean}\NormalTok{(sample_p)}

\NormalTok{    p_value <-}\StringTok{ }\KeywordTok{pnorm}\NormalTok{(media, }\DataTypeTok{mean =}\NormalTok{ lambda, }\DataTypeTok{sd =} \KeywordTok{sqrt}\NormalTok{(}\DecValTok{1}\OperatorTok{/}\NormalTok{n}\OperatorTok{*}\NormalTok{lambda))}
\NormalTok{    pvalue[i] <-}\StringTok{ }\NormalTok{p_value}

\NormalTok{    medias_v[i] <-}\StringTok{ }\NormalTok{media}

    \ControlFlowTok{if}\NormalTok{ (pvalue[i] }\OperatorTok{>=}\StringTok{ }\FloatTok{0.05}\NormalTok{) \{}
\NormalTok{      aceptance[i] <-}\StringTok{ }\DecValTok{1}
\NormalTok{    \}}
    \ControlFlowTok{else}\NormalTok{\{}
\NormalTok{      aceptance[i] <-}\StringTok{ }\DecValTok{0}
\NormalTok{    \}}
\NormalTok{  \}}

\NormalTok{  poder_}\DecValTok{1}\NormalTok{ <-}\StringTok{ }\KeywordTok{power.t.test}\NormalTok{(}\DataTypeTok{n =}\NormalTok{ n, }\DataTypeTok{delta =}\NormalTok{ poder_dif[}\DecValTok{1}\NormalTok{],}
                          \DataTypeTok{sd =} \KeywordTok{sqrt}\NormalTok{(}\DecValTok{1}\OperatorTok{/}\NormalTok{n}\OperatorTok{*}\NormalTok{lambda),}
                          \DataTypeTok{sig.level =} \FloatTok{.05}\NormalTok{,}
                          \DataTypeTok{alternative =} \StringTok{"two.sided"}\NormalTok{,}
                          \DataTypeTok{type =} \StringTok{"one.sample"}\NormalTok{)}
\NormalTok{  power_}\DecValTok{1}\NormalTok{[}\DecValTok{1}\NormalTok{] <-}\StringTok{ }\NormalTok{(}\KeywordTok{as.numeric}\NormalTok{(}\KeywordTok{unlist}\NormalTok{(poder_}\DecValTok{1}\NormalTok{[}\DecValTok{5}\NormalTok{])))}

\NormalTok{  poder_}\DecValTok{2}\NormalTok{ <-}\StringTok{ }\KeywordTok{power.t.test}\NormalTok{(}\DataTypeTok{n =}\NormalTok{ n, }\DataTypeTok{delta =}\NormalTok{ poder_dif[}\DecValTok{2}\NormalTok{],}
                          \DataTypeTok{sd =} \KeywordTok{sqrt}\NormalTok{(}\DecValTok{1}\OperatorTok{/}\NormalTok{n}\OperatorTok{*}\NormalTok{lambda),}
                          \DataTypeTok{sig.level =} \FloatTok{.05}\NormalTok{,}
                          \DataTypeTok{alternative =} \StringTok{"one.sided"}\NormalTok{,}
                          \DataTypeTok{type =} \StringTok{"one.sample"}\NormalTok{)}
\NormalTok{  power_}\DecValTok{1}\NormalTok{[}\DecValTok{2}\NormalTok{] <-}\StringTok{ }\KeywordTok{as.numeric}\NormalTok{(}\KeywordTok{unlist}\NormalTok{(poder_}\DecValTok{2}\NormalTok{[}\DecValTok{5}\NormalTok{]))}

\NormalTok{  poder_}\DecValTok{3}\NormalTok{ <-}\StringTok{ }\KeywordTok{power.t.test}\NormalTok{(}\DataTypeTok{n =}\NormalTok{ n, }\DataTypeTok{delta =}\NormalTok{ poder_dif[}\DecValTok{3}\NormalTok{],}
                          \DataTypeTok{sd =} \KeywordTok{sqrt}\NormalTok{(}\DecValTok{1}\OperatorTok{/}\NormalTok{n}\OperatorTok{*}\NormalTok{lambda),}
                          \DataTypeTok{sig.level =} \FloatTok{.05}\NormalTok{,}
                          \DataTypeTok{alternative =} \StringTok{"one.sided"}\NormalTok{,}
                          \DataTypeTok{type =} \StringTok{"one.sample"}\NormalTok{)}
\NormalTok{  power_}\DecValTok{1}\NormalTok{[}\DecValTok{3}\NormalTok{] <-}\StringTok{ }\KeywordTok{as.numeric}\NormalTok{(}\KeywordTok{unlist}\NormalTok{(poder_}\DecValTok{3}\NormalTok{[}\DecValTok{5}\NormalTok{]))}

\NormalTok{  poder_}\DecValTok{4}\NormalTok{ <-}\StringTok{ }\KeywordTok{power.t.test}\NormalTok{(}\DataTypeTok{n =}\NormalTok{ n, }\DataTypeTok{delta =}\NormalTok{ poder_dif[}\DecValTok{4}\NormalTok{],}
                          \DataTypeTok{sd =} \KeywordTok{sqrt}\NormalTok{(}\DecValTok{1}\OperatorTok{/}\NormalTok{n}\OperatorTok{*}\NormalTok{lambda),}
                          \DataTypeTok{sig.level =} \FloatTok{.05}\NormalTok{,}
                          \DataTypeTok{alternative =} \StringTok{"one.sided"}\NormalTok{,}
                          \DataTypeTok{type =} \StringTok{"one.sample"}\NormalTok{)}
\NormalTok{  power_}\DecValTok{1}\NormalTok{[}\DecValTok{4}\NormalTok{] <-}\StringTok{ }\KeywordTok{as.numeric}\NormalTok{(}\KeywordTok{unlist}\NormalTok{(poder_}\DecValTok{4}\NormalTok{[}\DecValTok{5}\NormalTok{]))}

\NormalTok{  poder_}\DecValTok{5}\NormalTok{ <-}\StringTok{ }\KeywordTok{power.t.test}\NormalTok{(}\DataTypeTok{n =}\NormalTok{ n, }\DataTypeTok{delta =}\NormalTok{ poder_dif[}\DecValTok{5}\NormalTok{],}
                          \DataTypeTok{sd =} \KeywordTok{sqrt}\NormalTok{(}\DecValTok{1}\OperatorTok{/}\NormalTok{n}\OperatorTok{*}\NormalTok{lambda),}
                          \DataTypeTok{sig.level =} \FloatTok{.05}\NormalTok{,}
                          \DataTypeTok{alternative =} \StringTok{"one.sided"}\NormalTok{,}
                          \DataTypeTok{type =} \StringTok{"one.sample"}\NormalTok{)}
\NormalTok{  power_}\DecValTok{1}\NormalTok{[}\DecValTok{5}\NormalTok{] <-}\StringTok{ }\KeywordTok{as.numeric}\NormalTok{(}\KeywordTok{unlist}\NormalTok{(poder_}\DecValTok{5}\NormalTok{[}\DecValTok{5}\NormalTok{]))}




\NormalTok{  results <-}\StringTok{ }\KeywordTok{list}\NormalTok{(}\DataTypeTok{pvalor =}\NormalTok{ pvalue, }\DataTypeTok{aceita =}\NormalTok{ aceptance,}
               \DataTypeTok{media =}\NormalTok{ medias_v, }\DataTypeTok{poder =}\NormalTok{ power_}\DecValTok{1}\NormalTok{)}

  \KeywordTok{return}\NormalTok{(results)}

\NormalTok{\}}


\NormalTok{a <-}\StringTok{ }\KeywordTok{po_sample}\NormalTok{(R, lambda, }\DecValTok{10}\NormalTok{)}
\NormalTok{b <-}\StringTok{ }\KeywordTok{po_sample}\NormalTok{(R, lambda, }\DecValTok{30}\NormalTok{)}
\NormalTok{c <-}\StringTok{ }\KeywordTok{po_sample}\NormalTok{(R, lambda, }\DecValTok{75}\NormalTok{)}
\NormalTok{d <-}\StringTok{ }\KeywordTok{po_sample}\NormalTok{(R, lambda, }\DecValTok{100}\NormalTok{)}
\end{Highlighting}
\end{Shaded}

\hypertarget{tabelas}{%
\subsection{Tabelas}\label{tabelas}}

~

Na tabela subsequente, é possível notar a tendência esperada, conforme o
o tamanho \(n\) da amostra aumenta, mais próximo de 95\% fica a taxa de
aceitação de \(H_0\). Este resultado é esperado pois, pela definição do
teste de hipótese estatístico, ao fixarmos \(\alpha = 0.05\) espera-se
que em 95\% dos casos testados sejam em direção a aceitação de \(H_0\)
se a suposição de normalidade dos dados realmente é válida.

\begin{Shaded}
\begin{Highlighting}[]
\NormalTok{freqa <-}\StringTok{ }\KeywordTok{table}\NormalTok{(a}\OperatorTok{$}\NormalTok{aceita) }\OperatorTok{*}\StringTok{ }\DecValTok{100} \OperatorTok{/}\StringTok{ }\KeywordTok{sum}\NormalTok{(}\KeywordTok{table}\NormalTok{(a}\OperatorTok{$}\NormalTok{aceita))}
\NormalTok{freqb <-}\StringTok{ }\KeywordTok{table}\NormalTok{(b}\OperatorTok{$}\NormalTok{aceita) }\OperatorTok{*}\StringTok{ }\DecValTok{100} \OperatorTok{/}\StringTok{ }\KeywordTok{sum}\NormalTok{(}\KeywordTok{table}\NormalTok{(b}\OperatorTok{$}\NormalTok{aceita))}
\NormalTok{freqc <-}\StringTok{ }\KeywordTok{table}\NormalTok{(c}\OperatorTok{$}\NormalTok{aceita) }\OperatorTok{*}\StringTok{ }\DecValTok{100} \OperatorTok{/}\StringTok{ }\KeywordTok{sum}\NormalTok{(}\KeywordTok{table}\NormalTok{(c}\OperatorTok{$}\NormalTok{aceita))}
\NormalTok{freqd <-}\StringTok{ }\KeywordTok{table}\NormalTok{(d}\OperatorTok{$}\NormalTok{aceita) }\OperatorTok{*}\StringTok{ }\DecValTok{100} \OperatorTok{/}\StringTok{ }\KeywordTok{sum}\NormalTok{(}\KeywordTok{table}\NormalTok{(d}\OperatorTok{$}\NormalTok{aceita))}

\NormalTok{tabela <-}\StringTok{ }\KeywordTok{rbind}\NormalTok{(}\StringTok{"Amostra n = 10"}\NormalTok{ =}\StringTok{ }\NormalTok{freqa, }\StringTok{"Amostra n = 30"}\NormalTok{ =}\StringTok{ }\NormalTok{freqb,}
                \StringTok{"Amostra n = 75"}\NormalTok{ =}\StringTok{ }\NormalTok{freqc, }\StringTok{"Amostra n = 100"}\NormalTok{ =}\StringTok{ }\NormalTok{freqd)}
\NormalTok{tabela_}\DecValTok{1}\NormalTok{ <-}\StringTok{ }\KeywordTok{data.frame}\NormalTok{(tabela)}

\NormalTok{knitr}\OperatorTok{::}\KeywordTok{kable}\NormalTok{(tabela_}\DecValTok{1}\NormalTok{, }\DataTypeTok{booktabs =}\NormalTok{ T, }\DataTypeTok{align =} \StringTok{"c"}\NormalTok{,}
             \DataTypeTok{caption =} \StringTok{"Tabela de Erro Tipo I (em %)"}\NormalTok{,}
             \DataTypeTok{col.names =} \KeywordTok{c}\NormalTok{(}\StringTok{"Rejeita Hipótese Nula"}\NormalTok{, }\StringTok{"Não Rejeita Hipótese Nula"}\NormalTok{),}
             \DataTypeTok{format =} \StringTok{"latex"}\NormalTok{, }\DataTypeTok{escape =}\NormalTok{ F) }\OperatorTok
\KeywordTok{kable_styling}\NormalTok{(}\DataTypeTok{position =} \StringTok{"center"}\NormalTok{, }\DataTypeTok{latex_options =} \KeywordTok{c}\NormalTok{(}\StringTok{"hold_position"}\NormalTok{))}
\end{Highlighting}
\end{Shaded}

\textbackslash{}begin\{table\}{[}!h{]}

\textbackslash{}caption\{\label{tab:unnamed-chunk-8}Tabela de Erro Tipo
I (em \%)\} \centering

\begin{tabular}[t]{lcc}
\toprule
  & Rejeita Hipótese Nula & Não Rejeita Hipótese Nula\\
\midrule
Amostra n = 10 & 4.07 & 95.93\\
Amostra n = 30 & 4.93 & 95.07\\
Amostra n = 75 & 3.79 & 96.21\\
Amostra n = 100 & 4.76 & 95.24\\
\bottomrule
\end{tabular}

\textbackslash{}end\{table\}

Agora, partindo para análise do poder, notamos outro resultado esperado
na tabela seguinte, a tendência apresentada é apresentada pelo gráfico
da seguinte forma: conforme cresce a diferença entre o valor testado do
paramêtro (\(\lambda = 2\)) e/ou cresce o tamanho da amostra maior é o
poder do teste, ou seja, torna-se cada vez mais sensível aos desvios do
valor de \(\hat{\lambda}_{EMV}\).

\begin{Shaded}
\begin{Highlighting}[]
\NormalTok{tabela_}\DecValTok{2}\NormalTok{ <-}\StringTok{ }\KeywordTok{rbind}\NormalTok{(}\StringTok{"Amostra n = 10"}\NormalTok{ =}\StringTok{ }\NormalTok{a}\OperatorTok{$}\NormalTok{poder, }\StringTok{"Amostra n = 30"}\NormalTok{ =}\StringTok{ }\NormalTok{b}\OperatorTok{$}\NormalTok{poder,}
                  \StringTok{"Amostra n = 75"}\NormalTok{ =}\StringTok{ }\NormalTok{c}\OperatorTok{$}\NormalTok{poder, }\StringTok{"Amostra n = 100"}\NormalTok{ =}\StringTok{ }\NormalTok{d}\OperatorTok{$}\NormalTok{poder)}
\NormalTok{p <-}\StringTok{ }\KeywordTok{seq}\NormalTok{(}\FloatTok{2.2}\NormalTok{, }\DecValTok{4}\NormalTok{, }\DataTypeTok{length.out =} \DecValTok{5}\NormalTok{)}
\NormalTok{df2 <-}\StringTok{ }\KeywordTok{data.frame}\NormalTok{(}\KeywordTok{round}\NormalTok{(tabela_}\DecValTok{2}\NormalTok{, }\DecValTok{2}\NormalTok{))}

\NormalTok{knitr}\OperatorTok{::}\KeywordTok{kable}\NormalTok{(df2, }\DataTypeTok{booktabs =}\NormalTok{ T, }\DataTypeTok{align =} \StringTok{"c"}\NormalTok{,}
             \DataTypeTok{caption =} \StringTok{"Tabela do Poder do Teste $}\CharTok{\textbackslash{}\textbackslash{}}\StringTok{lambda }\CharTok{\textbackslash{}\textbackslash{}}\StringTok{in[2,2;4]$"}\NormalTok{,}
             \DataTypeTok{col.names =} \KeywordTok{c}\NormalTok{(}\KeywordTok{paste}\NormalTok{(}\StringTok{"$}\CharTok{\textbackslash{}\textbackslash{}}\StringTok{lambda =$"}\NormalTok{, p[}\DecValTok{1}\NormalTok{]),}
                           \KeywordTok{paste}\NormalTok{(}\StringTok{"$}\CharTok{\textbackslash{}\textbackslash{}}\StringTok{lambda =$"}\NormalTok{, p[}\DecValTok{2}\NormalTok{]),}
                           \KeywordTok{paste}\NormalTok{(}\StringTok{"$}\CharTok{\textbackslash{}\textbackslash{}}\StringTok{lambda =$"}\NormalTok{, p[}\DecValTok{3}\NormalTok{]),}
                           \KeywordTok{paste}\NormalTok{(}\StringTok{"$}\CharTok{\textbackslash{}\textbackslash{}}\StringTok{lambda =$"}\NormalTok{, p[}\DecValTok{4}\NormalTok{]),}
                           \KeywordTok{paste}\NormalTok{(}\StringTok{"$}\CharTok{\textbackslash{}\textbackslash{}}\StringTok{lambda =$"}\NormalTok{, p[}\DecValTok{5}\NormalTok{])),}
             \DataTypeTok{format =} \StringTok{"latex"}\NormalTok{, }\DataTypeTok{escape =}\NormalTok{ F) }\OperatorTok
\KeywordTok{kable_styling}\NormalTok{(}\DataTypeTok{position =} \StringTok{"center"}\NormalTok{, }\DataTypeTok{latex_options =} \KeywordTok{c}\NormalTok{(}\StringTok{"hold_position"}\NormalTok{))}
\end{Highlighting}
\end{Shaded}

\begin{table}[!h]

\caption{\label{tab:unnamed-chunk-9}Tabela do Poder do Teste $\lambda \in[2,2;4]$}
\centering
\begin{tabular}[t]{lccccc}
\toprule
  & $\lambda =$ 2.2 & $\lambda =$ 2.65 & $\lambda =$ 3.1 & $\lambda =$ 3.55 & $\lambda =$ 4\\
\midrule
Amostra n = 10 & 0.24 & 0.99 & 1 & 1 & 1\\
Amostra n = 30 & 0.98 & 1.00 & 1 & 1 & 1\\
Amostra n = 75 & 1.00 & 1.00 & 1 & 1 & 1\\
Amostra n = 100 & 1.00 & 1.00 & 1 & 1 & 1\\
\bottomrule
\end{tabular}
\end{table}
\newpage

\hypertarget{conclusuxe3o}{%
\section{Conclusão}\label{conclusuxe3o}}

Após a compreensão e resolução, houve um notório aumento em relação a
percepção dos métodos de amostragem e geração de amostras por método
Monte Carlo pelos elaboradores deste trabalho, concluindo que o mesmo
foi fundamental para a melhora da visualização e entendimento dos
resultados esperados de cada processo formulado durante a disciplina. Ao
longo do trabalho, percebemos que poderíamos resolver as questões de
diversas formas, como no caso da estimação, por exemplo. Pelo conteúdo
da disciplina ser bem amplo, podemos associar a diversas outras
disciplinas.

\newpage

\hypertarget{apoio}{%
\section{Apoio}\label{apoio}}

Para elaborar o trabalho, o número 2021 foi selecionado como seed e,
para os exercícios 2 e 3, 10000 foi a quantidade de amostras geradas
para as estimações.

\end{document}
