\PassOptionsToPackage{unicode=true}{hyperref} % options for packages loaded elsewhere
\PassOptionsToPackage{hyphens}{url}
%
\documentclass[11pt,]{article}
\usepackage{lmodern}
\usepackage{amssymb,amsmath}
\usepackage{ifxetex,ifluatex}
\usepackage{fixltx2e} % provides \textsubscript
\ifnum 0\ifxetex 1\fi\ifluatex 1\fi=0 % if pdftex
  \usepackage[T1]{fontenc}
  \usepackage[utf8]{inputenc}
  \usepackage{textcomp} % provides euro and other symbols
\else % if luatex or xelatex
  \usepackage{unicode-math}
  \defaultfontfeatures{Ligatures=TeX,Scale=MatchLowercase}
\fi
% use upquote if available, for straight quotes in verbatim environments
\IfFileExists{upquote.sty}{\usepackage{upquote}}{}
% use microtype if available
\IfFileExists{microtype.sty}{%
\usepackage[]{microtype}
\UseMicrotypeSet[protrusion]{basicmath} % disable protrusion for tt fonts
}{}
\IfFileExists{parskip.sty}{%
\usepackage{parskip}
}{% else
\setlength{\parindent}{0pt}
\setlength{\parskip}{6pt plus 2pt minus 1pt}
}
\usepackage{hyperref}
\hypersetup{
            pdftitle={Exercício Aula 2},
            pdfauthor={Diego G. de Paulo (10857040); Vitor Gratiere Torres (10284952)},
            pdfborder={0 0 0},
            breaklinks=true}
\urlstyle{same}  % don't use monospace font for urls
\usepackage[margin=1in]{geometry}
\usepackage{color}
\usepackage{fancyvrb}
\newcommand{\VerbBar}{|}
\newcommand{\VERB}{\Verb[commandchars=\\\{\}]}
\DefineVerbatimEnvironment{Highlighting}{Verbatim}{commandchars=\\\{\}}
% Add ',fontsize=\small' for more characters per line
\usepackage{framed}
\definecolor{shadecolor}{RGB}{248,248,248}
\newenvironment{Shaded}{\begin{snugshade}}{\end{snugshade}}
\newcommand{\AlertTok}[1]{\textcolor[rgb]{0.94,0.16,0.16}{#1}}
\newcommand{\AnnotationTok}[1]{\textcolor[rgb]{0.56,0.35,0.01}{\textbf{\textit{#1}}}}
\newcommand{\AttributeTok}[1]{\textcolor[rgb]{0.77,0.63,0.00}{#1}}
\newcommand{\BaseNTok}[1]{\textcolor[rgb]{0.00,0.00,0.81}{#1}}
\newcommand{\BuiltInTok}[1]{#1}
\newcommand{\CharTok}[1]{\textcolor[rgb]{0.31,0.60,0.02}{#1}}
\newcommand{\CommentTok}[1]{\textcolor[rgb]{0.56,0.35,0.01}{\textit{#1}}}
\newcommand{\CommentVarTok}[1]{\textcolor[rgb]{0.56,0.35,0.01}{\textbf{\textit{#1}}}}
\newcommand{\ConstantTok}[1]{\textcolor[rgb]{0.00,0.00,0.00}{#1}}
\newcommand{\ControlFlowTok}[1]{\textcolor[rgb]{0.13,0.29,0.53}{\textbf{#1}}}
\newcommand{\DataTypeTok}[1]{\textcolor[rgb]{0.13,0.29,0.53}{#1}}
\newcommand{\DecValTok}[1]{\textcolor[rgb]{0.00,0.00,0.81}{#1}}
\newcommand{\DocumentationTok}[1]{\textcolor[rgb]{0.56,0.35,0.01}{\textbf{\textit{#1}}}}
\newcommand{\ErrorTok}[1]{\textcolor[rgb]{0.64,0.00,0.00}{\textbf{#1}}}
\newcommand{\ExtensionTok}[1]{#1}
\newcommand{\FloatTok}[1]{\textcolor[rgb]{0.00,0.00,0.81}{#1}}
\newcommand{\FunctionTok}[1]{\textcolor[rgb]{0.00,0.00,0.00}{#1}}
\newcommand{\ImportTok}[1]{#1}
\newcommand{\InformationTok}[1]{\textcolor[rgb]{0.56,0.35,0.01}{\textbf{\textit{#1}}}}
\newcommand{\KeywordTok}[1]{\textcolor[rgb]{0.13,0.29,0.53}{\textbf{#1}}}
\newcommand{\NormalTok}[1]{#1}
\newcommand{\OperatorTok}[1]{\textcolor[rgb]{0.81,0.36,0.00}{\textbf{#1}}}
\newcommand{\OtherTok}[1]{\textcolor[rgb]{0.56,0.35,0.01}{#1}}
\newcommand{\PreprocessorTok}[1]{\textcolor[rgb]{0.56,0.35,0.01}{\textit{#1}}}
\newcommand{\RegionMarkerTok}[1]{#1}
\newcommand{\SpecialCharTok}[1]{\textcolor[rgb]{0.00,0.00,0.00}{#1}}
\newcommand{\SpecialStringTok}[1]{\textcolor[rgb]{0.31,0.60,0.02}{#1}}
\newcommand{\StringTok}[1]{\textcolor[rgb]{0.31,0.60,0.02}{#1}}
\newcommand{\VariableTok}[1]{\textcolor[rgb]{0.00,0.00,0.00}{#1}}
\newcommand{\VerbatimStringTok}[1]{\textcolor[rgb]{0.31,0.60,0.02}{#1}}
\newcommand{\WarningTok}[1]{\textcolor[rgb]{0.56,0.35,0.01}{\textbf{\textit{#1}}}}
\usepackage{graphicx,grffile}
\makeatletter
\def\maxwidth{\ifdim\Gin@nat@width>\linewidth\linewidth\else\Gin@nat@width\fi}
\def\maxheight{\ifdim\Gin@nat@height>\textheight\textheight\else\Gin@nat@height\fi}
\makeatother
% Scale images if necessary, so that they will not overflow the page
% margins by default, and it is still possible to overwrite the defaults
% using explicit options in \includegraphics[width, height, ...]{}
\setkeys{Gin}{width=\maxwidth,height=\maxheight,keepaspectratio}
\setlength{\emergencystretch}{3em}  % prevent overfull lines
\providecommand{\tightlist}{%
  \setlength{\itemsep}{0pt}\setlength{\parskip}{0pt}}
\setcounter{secnumdepth}{0}
% Redefines (sub)paragraphs to behave more like sections
\ifx\paragraph\undefined\else
\let\oldparagraph\paragraph
\renewcommand{\paragraph}[1]{\oldparagraph{#1}\mbox{}}
\fi
\ifx\subparagraph\undefined\else
\let\oldsubparagraph\subparagraph
\renewcommand{\subparagraph}[1]{\oldsubparagraph{#1}\mbox{}}
\fi

% set default figure placement to htbp
\makeatletter
\def\fps@figure{htbp}
\makeatother

\usepackage{enumerate}
\usepackage{hyperref}
\usepackage{booktabs}

\title{Exercício Aula 2}
\author{Diego G. de Paulo (10857040) \and Vitor Gratiere Torres (10284952)}
\date{27/04/2020}

\begin{document}
\maketitle

\hypertarget{ex-1}{%
\section{EX 1}\label{ex-1}}

Motivação: O intuito deste exercício é gerar uma amostra
pseudo-aleatória de \(f(x)\) dada por
\(f(x) \propto q(x) = e^{\frac{-|x|^3}{3}}\).

Para gerar tal amostra, foi selecionado o método da rejeição. Este
método é descrito por:

A seleção de uma variável aleatória \(Y\), com função de densidade dada
por \(g(y)\) amostrável. Além disso, há a suposição:

\begin{itemize}
\tightlist
\item
  \(\frac{f(x)}{g(x)} \leq m\), \(1 \leq m < \infty\)
\end{itemize}

E, por recomendação, toma-se
\(m = max_x\left( \frac{f(x)}{g(x)} \right)\)

Para o exercício em questão, seleciona-se \(Y\), tal que
\(Y \sim Laplace(0,1)\) que tem a função de probabilidade dada por:
\(g(y) = \frac{1}{2} e^{-|y|}, y \in \mathbb{R}\). Para obter m, tem-se:
\[m = max_x\left( \frac{f(x)}{g(x)} \right) \iff \frac{d\left(\frac{e^{\frac{-|x|^3}{3}}}{\frac{1}{2} e^{-|x|}}  \right)}{dx} = 0\]
Calculando a derivada de \(\frac{f(x)}{g(x)}\):

\[
\frac{d\left(\frac{f(x)}{g(x)}
\right)}{dx} =
\frac{d\left(\frac{e^{\frac{-|x|^3}{3}}}{\frac{1}{2} e^{-|x|}}  \right)}{dx} =
\frac{2e^{\frac{-|x|^3+3|x|}{3}}x(-x^2+1)}{|x|}
\]

Igualando a zero:

\[\frac{2e^{\frac{-|x|^3+3|x|}{3}}x(-x^2+1)}{|x|} = 0\] Como solução
para esta equação tem-se: \(x = -1, 0, 1\), afim de não postergar o
cálculo e partir para o gerador de amostra pseudo-aleatória,
seleciona-se, dos pontos críticos, apenas os pontos de máximo em
\(x = -1, 1\). Sendo assim obtém-se:
\(m = max_x\left( \frac{f(x)}{g(x)} \right) = 2e^{\frac{2}{3}}\).
Finalizada a etapa de seleção das variáveis, segue a apliacação das
etapas:

\begin{itemize}
\tightlist
\item
  1º: Gerar uma amostra de \(Y\)
\item
  2º: Gerar \(u \sim U(0,1)\)
\item
  3º: Se \(u \leq \frac{f(y)}{mg(y)}\) faça \(x = y\), Caso contrário
  retornar ao primeiro passo.
\item
  4º: Repita os passos anteriores até obter n observações necessárias.
\end{itemize}

Abaixo, para melhor visualização dos resultados obtidos matematicamente,
estão exibidos os gráficos das funções \(f(x)\), \(g(y)\), assim como os
gráficos de \(\frac{f(x)}{g(x)}\), para que os pontos critícos possam
ser observados, e o gráfico sobreposto de \(f(x)\) e \(mg(x)\), para
notar o envolapamento de \(f(x)\) por \(mg(x)\).

\includegraphics{trab1_files/figure-latex/unnamed-chunk-2-1.pdf}

A seguir, é possível observar histogramas e boxplots para cada tamanho
de amostra \(n = (50, 100, 400)\) das amostras pseudo-aleatórias geradas
da \(f(x)\). É possível notar, observando o gráfico, que a medida em que
se aumenta o tamanho da amostra mais próximo da simetria observada na
função, dispõe-se a amostra.

\includegraphics{trab1_files/figure-latex/unnamed-chunk-3-1.pdf}

\hypertarget{ex-2}{%
\section{EX 2}\label{ex-2}}

\hypertarget{ex-3}{%
\section{EX 3}\label{ex-3}}

Motivação: Neste presente exercício busca-se trabalhar com a
distribuição poisson e estudar os Erros do Tipo I e II para esta
distribuição utilizando o Método de Monte Carlo.

Para facilitar a compreensão, será indicado alguns aspectos importantes
de tal distribuição. Dado \(X \sim Poisson(\lambda)\), \(X_1, ..., X_n\)
é uma amostra aleatória de \(X\) de tamanho \(n\).

\begin{itemize}
\item
  \(\mathbb{P}(X=k)=\frac{e^{-\lambda}\lambda^x}{x!}\), para
  \(\lambda > 0\) e \(x = 0, 1, 2, ...\);
\item
  \(\mathbb{E}(X) = \lambda\);
\item
  \(\text{Var}(X) = \lambda\);
\item
  \(\hat{\lambda}_{EMV} = \frac{1}{n}\sum_{i=1}^n X_i = \bar{X}\), em
  que \(k_i\) é a i-ésima observação amostral de \(X_i\);
\end{itemize}

Pela propriedade da eficiêcnia de um estimador de máxima
verossimilhança, tem-se o resultado:

\[\sqrt{n}(\hat{\lambda}_{EMV}-\lambda)\xrightarrow{_D}\mathcal{N}(0,\mathcal{IF}(\lambda)^{-1})\]
em que \(\mathcal{IF}(\lambda) = \frac{1}{\lambda}\)

Para o teste de hipótese:

\begin{itemize}
\tightlist
\item
  \(H_0:\) \(\lambda = 2\)
\item
  \(H_A:\) \(\lambda > 2\)
\end{itemize}

A Estatística do Teste \(T\) pode ser definida a partir do resultado da
efiência do estimador de máxima verossimilhança.

\[T = \sqrt{n}(\hat{\lambda}_{EMV}-2)\xrightarrow[\text{Sob } H_0]{D}\mathcal{N}\left(0,\frac{1}{2} \right)\]

\begin{Shaded}
\begin{Highlighting}[]
\NormalTok{r <-}\StringTok{ }\DecValTok{1000}
\NormalTok{lambda <-}\StringTok{ }\DecValTok{2}
\NormalTok{po_sample <-}\StringTok{ }\ControlFlowTok{function}\NormalTok{(r, lambda, n) \{}

\NormalTok{  pvalue <-}\StringTok{ }\KeywordTok{c}\NormalTok{()}
\NormalTok{  aceptance <-}\StringTok{ }\KeywordTok{c}\NormalTok{()}
\NormalTok{  medias_v <-}\StringTok{ }\KeywordTok{c}\NormalTok{()}
\NormalTok{  poder_dif <-}\StringTok{ }\KeywordTok{seq}\NormalTok{(}\FloatTok{2.2}\NormalTok{, }\DecValTok{4}\NormalTok{, }\DataTypeTok{length.out =} \DecValTok{5}\NormalTok{) }\OperatorTok{-}\StringTok{ }\DecValTok{2}
\NormalTok{  power_}\DecValTok{1}\NormalTok{ <-}\StringTok{ }\KeywordTok{c}\NormalTok{()}


  \ControlFlowTok{for}\NormalTok{ (i }\ControlFlowTok{in} \DecValTok{1}\OperatorTok{:}\NormalTok{r) \{}

\NormalTok{    sample_p <-}\StringTok{ }\KeywordTok{rpois}\NormalTok{(n, lambda)}
\NormalTok{    media <-}\StringTok{ }\KeywordTok{mean}\NormalTok{(sample_p)}

\NormalTok{    p_value <-}\StringTok{ }\KeywordTok{pnorm}\NormalTok{(media, }\DataTypeTok{mean =}\NormalTok{ lambda, }\DataTypeTok{sd =} \KeywordTok{sqrt}\NormalTok{(}\DecValTok{1} \OperatorTok{/}\StringTok{ }\NormalTok{n }\OperatorTok{*}\StringTok{ }\NormalTok{lambda))}
\NormalTok{    pvalue[i] <-}\StringTok{ }\NormalTok{p_value}

\NormalTok{    medias_v[i] <-}\StringTok{ }\NormalTok{media}

    \ControlFlowTok{if}\NormalTok{ (pvalue[i] }\OperatorTok{>=}\StringTok{ }\FloatTok{0.05}\NormalTok{) \{}
\NormalTok{      aceptance[i] <-}\StringTok{ }\DecValTok{1}
\NormalTok{    \}}
    \ControlFlowTok{else}\NormalTok{ \{}
\NormalTok{      aceptance[i] <-}\StringTok{ }\DecValTok{0}
\NormalTok{    \}}
\NormalTok{  \}}

\NormalTok{  poder_}\DecValTok{1}\NormalTok{ <-}\StringTok{ }\KeywordTok{power.t.test}\NormalTok{(}\DataTypeTok{n =}\NormalTok{ n, }\DataTypeTok{delta =}\NormalTok{ poder_dif[}\DecValTok{1}\NormalTok{],}
                          \DataTypeTok{sd =} \KeywordTok{sqrt}\NormalTok{(}\DecValTok{1} \OperatorTok{/}\StringTok{ }\NormalTok{n }\OperatorTok{*}\StringTok{ }\NormalTok{lambda),}
                          \DataTypeTok{sig.level =} \FloatTok{.05}\NormalTok{,}
                          \DataTypeTok{alternative =} \StringTok{"two.sided"}\NormalTok{,}
                          \DataTypeTok{type =} \StringTok{"one.sample"}\NormalTok{)}
\NormalTok{  power_}\DecValTok{1}\NormalTok{[}\DecValTok{1}\NormalTok{] <-}\StringTok{ }\NormalTok{(}\KeywordTok{as.numeric}\NormalTok{(}\KeywordTok{unlist}\NormalTok{(poder_}\DecValTok{1}\NormalTok{[}\DecValTok{5}\NormalTok{])))}

\NormalTok{  poder_}\DecValTok{2}\NormalTok{ <-}\StringTok{ }\KeywordTok{power.t.test}\NormalTok{(}\DataTypeTok{n =}\NormalTok{ n, }\DataTypeTok{delta =}\NormalTok{ poder_dif[}\DecValTok{2}\NormalTok{],}
                          \DataTypeTok{sd =} \KeywordTok{sqrt}\NormalTok{(}\DecValTok{1} \OperatorTok{/}\StringTok{ }\NormalTok{n }\OperatorTok{*}\StringTok{ }\NormalTok{lambda),}
                          \DataTypeTok{sig.level =} \FloatTok{.05}\NormalTok{,}
                          \DataTypeTok{alternative =} \StringTok{"one.sided"}\NormalTok{,}
                          \DataTypeTok{type =} \StringTok{"one.sample"}\NormalTok{)}
\NormalTok{  power_}\DecValTok{1}\NormalTok{[}\DecValTok{2}\NormalTok{] <-}\StringTok{ }\KeywordTok{as.numeric}\NormalTok{(}\KeywordTok{unlist}\NormalTok{(poder_}\DecValTok{2}\NormalTok{[}\DecValTok{5}\NormalTok{]))}

\NormalTok{  poder_}\DecValTok{3}\NormalTok{ <-}\StringTok{ }\KeywordTok{power.t.test}\NormalTok{(}\DataTypeTok{n =}\NormalTok{ n, }\DataTypeTok{delta =}\NormalTok{ poder_dif[}\DecValTok{3}\NormalTok{],}
                          \DataTypeTok{sd =} \KeywordTok{sqrt}\NormalTok{(}\DecValTok{1} \OperatorTok{/}\StringTok{ }\NormalTok{n }\OperatorTok{*}\StringTok{ }\NormalTok{lambda),}
                          \DataTypeTok{sig.level =} \FloatTok{.05}\NormalTok{,}
                          \DataTypeTok{alternative =} \StringTok{"one.sided"}\NormalTok{,}
                          \DataTypeTok{type =} \StringTok{"one.sample"}\NormalTok{)}
\NormalTok{  power_}\DecValTok{1}\NormalTok{[}\DecValTok{3}\NormalTok{] <-}\StringTok{ }\KeywordTok{as.numeric}\NormalTok{(}\KeywordTok{unlist}\NormalTok{(poder_}\DecValTok{3}\NormalTok{[}\DecValTok{5}\NormalTok{]))}

\NormalTok{  poder_}\DecValTok{4}\NormalTok{ <-}\StringTok{ }\KeywordTok{power.t.test}\NormalTok{(}\DataTypeTok{n =}\NormalTok{ n, }\DataTypeTok{delta =}\NormalTok{ poder_dif[}\DecValTok{4}\NormalTok{],}
                          \DataTypeTok{sd =} \KeywordTok{sqrt}\NormalTok{(}\DecValTok{1} \OperatorTok{/}\StringTok{ }\NormalTok{n }\OperatorTok{*}\StringTok{ }\NormalTok{lambda),}
                          \DataTypeTok{sig.level =} \FloatTok{.05}\NormalTok{,}
                          \DataTypeTok{alternative =} \StringTok{"one.sided"}\NormalTok{,}
                          \DataTypeTok{type =} \StringTok{"one.sample"}\NormalTok{)}
\NormalTok{  power_}\DecValTok{1}\NormalTok{[}\DecValTok{4}\NormalTok{] <-}\StringTok{ }\KeywordTok{as.numeric}\NormalTok{(}\KeywordTok{unlist}\NormalTok{(poder_}\DecValTok{4}\NormalTok{[}\DecValTok{5}\NormalTok{]))}

\NormalTok{  poder_}\DecValTok{5}\NormalTok{ <-}\StringTok{ }\KeywordTok{power.t.test}\NormalTok{(}\DataTypeTok{n =}\NormalTok{ n, }\DataTypeTok{delta =}\NormalTok{ poder_dif[}\DecValTok{5}\NormalTok{],}
                          \DataTypeTok{sd =} \KeywordTok{sqrt}\NormalTok{(}\DecValTok{1} \OperatorTok{/}\StringTok{ }\NormalTok{n }\OperatorTok{*}\StringTok{ }\NormalTok{lambda),}
                          \DataTypeTok{sig.level =} \FloatTok{.05}\NormalTok{,}
                          \DataTypeTok{alternative =} \StringTok{"one.sided"}\NormalTok{,}
                          \DataTypeTok{type =} \StringTok{"one.sample"}\NormalTok{)}
\NormalTok{  power_}\DecValTok{1}\NormalTok{[}\DecValTok{5}\NormalTok{] <-}\StringTok{ }\KeywordTok{as.numeric}\NormalTok{(}\KeywordTok{unlist}\NormalTok{(poder_}\DecValTok{5}\NormalTok{[}\DecValTok{5}\NormalTok{]))}

\NormalTok{  results <-}\StringTok{ }\KeywordTok{list}\NormalTok{(}\DataTypeTok{pvalor =}\NormalTok{ pvalue, }\DataTypeTok{aceita =}\NormalTok{ aceptance,}
                  \DataTypeTok{media =}\NormalTok{ medias_v, }\DataTypeTok{poder =}\NormalTok{ power_}\DecValTok{1}\NormalTok{)}

  \KeywordTok{return}\NormalTok{(results)}
\NormalTok{\}}

\NormalTok{a <-}\StringTok{ }\KeywordTok{po_sample}\NormalTok{(r, lambda, }\DecValTok{10}\NormalTok{)}
\NormalTok{b <-}\StringTok{ }\KeywordTok{po_sample}\NormalTok{(r, lambda, }\DecValTok{30}\NormalTok{)}
\NormalTok{c <-}\StringTok{ }\KeywordTok{po_sample}\NormalTok{(r, lambda, }\DecValTok{75}\NormalTok{)}
\NormalTok{d <-}\StringTok{ }\KeywordTok{po_sample}\NormalTok{(r, lambda, }\DecValTok{100}\NormalTok{)}
\end{Highlighting}
\end{Shaded}

Na tabela subsequente, é possível notar a tendência esperada, conforme o
o tamanho \(n\) da amostra aumenta, mais próximo de 95\% fica a taxa de
aceitação de \(H_0\). Este resultado é esperado pois, pela definição do
teste de hipótese estatístico, ao fixarmos \(\alpha = 0.05\) espera-se
que em 95\% dos casos testados sejam em direção a aceitação de \(H_0\)
se a suposição de normalidade dos dados realmente é válida.

\begin{Shaded}
\begin{Highlighting}[]
\NormalTok{freqa <-}\StringTok{ }\KeywordTok{table}\NormalTok{(a}\OperatorTok{$}\NormalTok{aceita) }\OperatorTok{*}\StringTok{ }\DecValTok{100} \OperatorTok{/}\StringTok{ }\KeywordTok{sum}\NormalTok{(}\KeywordTok{table}\NormalTok{(a}\OperatorTok{$}\NormalTok{aceita))}
\NormalTok{freqb <-}\StringTok{ }\KeywordTok{table}\NormalTok{(b}\OperatorTok{$}\NormalTok{aceita) }\OperatorTok{*}\StringTok{ }\DecValTok{100} \OperatorTok{/}\StringTok{ }\KeywordTok{sum}\NormalTok{(}\KeywordTok{table}\NormalTok{(b}\OperatorTok{$}\NormalTok{aceita))}
\NormalTok{freqc <-}\StringTok{ }\KeywordTok{table}\NormalTok{(c}\OperatorTok{$}\NormalTok{aceita) }\OperatorTok{*}\StringTok{ }\DecValTok{100} \OperatorTok{/}\StringTok{ }\KeywordTok{sum}\NormalTok{(}\KeywordTok{table}\NormalTok{(c}\OperatorTok{$}\NormalTok{aceita))}
\NormalTok{freqd <-}\StringTok{ }\KeywordTok{table}\NormalTok{(d}\OperatorTok{$}\NormalTok{aceita) }\OperatorTok{*}\StringTok{ }\DecValTok{100} \OperatorTok{/}\StringTok{ }\KeywordTok{sum}\NormalTok{(}\KeywordTok{table}\NormalTok{(d}\OperatorTok{$}\NormalTok{aceita))}

\NormalTok{tabela <-}\StringTok{ }\KeywordTok{rbind}\NormalTok{(}\StringTok{"Amostra n = 10"}\NormalTok{ =}\StringTok{ }\NormalTok{freqa, }\StringTok{"Amostra n = 30"}\NormalTok{ =}\StringTok{ }\NormalTok{freqb,}
                \StringTok{"Amostra n = 75"}\NormalTok{ =}\StringTok{ }\NormalTok{freqc, }\StringTok{"Amostra n = 100"}\NormalTok{ =}\StringTok{ }\NormalTok{freqd)}
\NormalTok{tabela_}\DecValTok{1}\NormalTok{ <-}\StringTok{ }\KeywordTok{data.frame}\NormalTok{(tabela)}

\NormalTok{knitr}\OperatorTok{::}\KeywordTok{kable}\NormalTok{(tabela_}\DecValTok{1}\NormalTok{, }\DataTypeTok{booktabs =}\NormalTok{ T, }\DataTypeTok{align =} \StringTok{"c"}\NormalTok{,}
             \DataTypeTok{caption =} \StringTok{"Tabela de Erro Tipo I (em %)"}\NormalTok{,}
             \DataTypeTok{col.names =} \KeywordTok{c}\NormalTok{(}\StringTok{"Rejeita Hipótese Nula"}\NormalTok{, }\StringTok{"Não Rejeita Hipótese Nula"}\NormalTok{),}
             \DataTypeTok{format =} \StringTok{"latex"}\NormalTok{, }\DataTypeTok{escape =}\NormalTok{ T) }\OperatorTok
\KeywordTok{kable_styling}\NormalTok{(}\DataTypeTok{position =} \StringTok{"center"}\NormalTok{, }\DataTypeTok{latex_options =} \KeywordTok{c}\NormalTok{(}\StringTok{"hold_position"}\NormalTok{))}
\end{Highlighting}
\end{Shaded}

\textbackslash{}begin\{table\}{[}!h{]}

\textbackslash{}caption\{\label{tab:unnamed-chunk-6}Tabela de Erro Tipo
I (em \%)\} \centering

\begin{tabular}[t]{lcc}
\toprule
  & Rejeita Hipótese Nula & Não Rejeita Hipótese Nula\\
\midrule
Amostra n = 10 & 3.1 & 96.9\\
Amostra n = 30 & 5.1 & 94.9\\
Amostra n = 75 & 4.1 & 95.9\\
Amostra n = 100 & 4.9 & 95.1\\
\bottomrule
\end{tabular}

\textbackslash{}end\{table\}

Agora, partindo para análise do poder, notamos outro resultado esperado
na tabela seguinte, a tendência apresentada é apresentada pelo gráfico
da seguinte forma: conforme cresce a diferença entre o valor testado do
paramêtro (\(\lambda = 2\)) e/ou cresce o tamanho da amostra maior é o
poder do teste, ou seja, torna-se cada vez mais sensível aos desvios do
valor de \(\hat{\lambda}_{EMV}\).

\begin{Shaded}
\begin{Highlighting}[]
\NormalTok{tabela_}\DecValTok{2}\NormalTok{ <-}\StringTok{ }\KeywordTok{rbind}\NormalTok{(}\StringTok{"Amostra n = 10"}\NormalTok{ =}\StringTok{ }\NormalTok{a}\OperatorTok{$}\NormalTok{poder, }\StringTok{"Amostra n = 30"}\NormalTok{ =}\StringTok{ }\NormalTok{b}\OperatorTok{$}\NormalTok{poder,}
                  \StringTok{"Amostra n = 75"}\NormalTok{ =}\StringTok{ }\NormalTok{c}\OperatorTok{$}\NormalTok{poder, }\StringTok{"Amostra n = 100"}\NormalTok{ =}\StringTok{ }\NormalTok{d}\OperatorTok{$}\NormalTok{poder)}
\NormalTok{p <-}\StringTok{ }\KeywordTok{seq}\NormalTok{(}\FloatTok{2.2}\NormalTok{, }\DecValTok{4}\NormalTok{, }\DataTypeTok{length.out =} \DecValTok{5}\NormalTok{)}
\NormalTok{df2 <-}\StringTok{ }\KeywordTok{data.frame}\NormalTok{(}\KeywordTok{round}\NormalTok{(tabela_}\DecValTok{2}\NormalTok{, }\DecValTok{2}\NormalTok{))}

\NormalTok{knitr}\OperatorTok{::}\KeywordTok{kable}\NormalTok{(df2, }\DataTypeTok{booktabs =}\NormalTok{ T, }\DataTypeTok{align =} \StringTok{"c"}\NormalTok{,}
             \DataTypeTok{caption =} \StringTok{"Tabela do Poder do Teste $}\CharTok{\textbackslash{}\textbackslash{}}\StringTok{lambda }\CharTok{\textbackslash{}\textbackslash{}}\StringTok{in[2,2;4]$"}\NormalTok{,}
             \DataTypeTok{col.names =} \KeywordTok{c}\NormalTok{(}\KeywordTok{paste}\NormalTok{(}\StringTok{"$}\CharTok{\textbackslash{}\textbackslash{}}\StringTok{lambda =$"}\NormalTok{, p[}\DecValTok{1}\NormalTok{]),}
                           \KeywordTok{paste}\NormalTok{(}\StringTok{"$}\CharTok{\textbackslash{}\textbackslash{}}\StringTok{lambda =$"}\NormalTok{, p[}\DecValTok{2}\NormalTok{]),}
                           \KeywordTok{paste}\NormalTok{(}\StringTok{"$}\CharTok{\textbackslash{}\textbackslash{}}\StringTok{lambda =$"}\NormalTok{, p[}\DecValTok{3}\NormalTok{]),}
                           \KeywordTok{paste}\NormalTok{(}\StringTok{"$}\CharTok{\textbackslash{}\textbackslash{}}\StringTok{lambda =$"}\NormalTok{, p[}\DecValTok{4}\NormalTok{]),}
                           \KeywordTok{paste}\NormalTok{(}\StringTok{"$}\CharTok{\textbackslash{}\textbackslash{}}\StringTok{lambda =$"}\NormalTok{, p[}\DecValTok{5}\NormalTok{])),}
             \DataTypeTok{format =} \StringTok{"latex"}\NormalTok{, }\DataTypeTok{escape =}\NormalTok{ F) }\OperatorTok
\KeywordTok{kable_styling}\NormalTok{(}\DataTypeTok{position =} \StringTok{"center"}\NormalTok{, }\DataTypeTok{latex_options =} \KeywordTok{c}\NormalTok{(}\StringTok{"hold_position"}\NormalTok{))}
\end{Highlighting}
\end{Shaded}

\begin{table}[!h]

\caption{\label{tab:unnamed-chunk-7}Tabela do Poder do Teste $\lambda \in[2,2;4]$}
\centering
\begin{tabular}[t]{lccccc}
\toprule
  & $\lambda =$ 2.2 & $\lambda =$ 2.65 & $\lambda =$ 3.1 & $\lambda =$ 3.55 & $\lambda =$ 4\\
\midrule
Amostra n = 10 & 0.24 & 0.99 & 1 & 1 & 1\\
Amostra n = 30 & 0.98 & 1.00 & 1 & 1 & 1\\
Amostra n = 75 & 1.00 & 1.00 & 1 & 1 & 1\\
Amostra n = 100 & 1.00 & 1.00 & 1 & 1 & 1\\
\bottomrule
\end{tabular}
\end{table}

\end{document}
