\PassOptionsToPackage{unicode=true}{hyperref} % options for packages loaded elsewhere
\PassOptionsToPackage{hyphens}{url}
%
\documentclass[11pt,]{article}
\usepackage{lmodern}
\usepackage{amssymb,amsmath}
\usepackage{ifxetex,ifluatex}
\usepackage{fixltx2e} % provides \textsubscript
\ifnum 0\ifxetex 1\fi\ifluatex 1\fi=0 % if pdftex
  \usepackage[T1]{fontenc}
  \usepackage[utf8]{inputenc}
  \usepackage{textcomp} % provides euro and other symbols
\else % if luatex or xelatex
  \usepackage{unicode-math}
  \defaultfontfeatures{Ligatures=TeX,Scale=MatchLowercase}
\fi
% use upquote if available, for straight quotes in verbatim environments
\IfFileExists{upquote.sty}{\usepackage{upquote}}{}
% use microtype if available
\IfFileExists{microtype.sty}{%
\usepackage[]{microtype}
\UseMicrotypeSet[protrusion]{basicmath} % disable protrusion for tt fonts
}{}
\IfFileExists{parskip.sty}{%
\usepackage{parskip}
}{% else
\setlength{\parindent}{0pt}
\setlength{\parskip}{6pt plus 2pt minus 1pt}
}
\usepackage{hyperref}
\hypersetup{
            pdftitle={Trabalho 2 - SME0806 - Estatística Computacional},
            pdfauthor={Diego G. de Paulo (10857040); Bruno H.da S Justino (11031621); Douglas S. Souza (10733820); Caio H. M. Schiavo (11810602); Vitor Gratiere Torres (10284952)},
            pdfborder={0 0 0},
            breaklinks=true}
\urlstyle{same}  % don't use monospace font for urls
\usepackage[margin=1in]{geometry}
\usepackage{color}
\usepackage{fancyvrb}
\newcommand{\VerbBar}{|}
\newcommand{\VERB}{\Verb[commandchars=\\\{\}]}
\DefineVerbatimEnvironment{Highlighting}{Verbatim}{commandchars=\\\{\}}
% Add ',fontsize=\small' for more characters per line
\usepackage{framed}
\definecolor{shadecolor}{RGB}{248,248,248}
\newenvironment{Shaded}{\begin{snugshade}}{\end{snugshade}}
\newcommand{\AlertTok}[1]{\textcolor[rgb]{0.94,0.16,0.16}{#1}}
\newcommand{\AnnotationTok}[1]{\textcolor[rgb]{0.56,0.35,0.01}{\textbf{\textit{#1}}}}
\newcommand{\AttributeTok}[1]{\textcolor[rgb]{0.77,0.63,0.00}{#1}}
\newcommand{\BaseNTok}[1]{\textcolor[rgb]{0.00,0.00,0.81}{#1}}
\newcommand{\BuiltInTok}[1]{#1}
\newcommand{\CharTok}[1]{\textcolor[rgb]{0.31,0.60,0.02}{#1}}
\newcommand{\CommentTok}[1]{\textcolor[rgb]{0.56,0.35,0.01}{\textit{#1}}}
\newcommand{\CommentVarTok}[1]{\textcolor[rgb]{0.56,0.35,0.01}{\textbf{\textit{#1}}}}
\newcommand{\ConstantTok}[1]{\textcolor[rgb]{0.00,0.00,0.00}{#1}}
\newcommand{\ControlFlowTok}[1]{\textcolor[rgb]{0.13,0.29,0.53}{\textbf{#1}}}
\newcommand{\DataTypeTok}[1]{\textcolor[rgb]{0.13,0.29,0.53}{#1}}
\newcommand{\DecValTok}[1]{\textcolor[rgb]{0.00,0.00,0.81}{#1}}
\newcommand{\DocumentationTok}[1]{\textcolor[rgb]{0.56,0.35,0.01}{\textbf{\textit{#1}}}}
\newcommand{\ErrorTok}[1]{\textcolor[rgb]{0.64,0.00,0.00}{\textbf{#1}}}
\newcommand{\ExtensionTok}[1]{#1}
\newcommand{\FloatTok}[1]{\textcolor[rgb]{0.00,0.00,0.81}{#1}}
\newcommand{\FunctionTok}[1]{\textcolor[rgb]{0.00,0.00,0.00}{#1}}
\newcommand{\ImportTok}[1]{#1}
\newcommand{\InformationTok}[1]{\textcolor[rgb]{0.56,0.35,0.01}{\textbf{\textit{#1}}}}
\newcommand{\KeywordTok}[1]{\textcolor[rgb]{0.13,0.29,0.53}{\textbf{#1}}}
\newcommand{\NormalTok}[1]{#1}
\newcommand{\OperatorTok}[1]{\textcolor[rgb]{0.81,0.36,0.00}{\textbf{#1}}}
\newcommand{\OtherTok}[1]{\textcolor[rgb]{0.56,0.35,0.01}{#1}}
\newcommand{\PreprocessorTok}[1]{\textcolor[rgb]{0.56,0.35,0.01}{\textit{#1}}}
\newcommand{\RegionMarkerTok}[1]{#1}
\newcommand{\SpecialCharTok}[1]{\textcolor[rgb]{0.00,0.00,0.00}{#1}}
\newcommand{\SpecialStringTok}[1]{\textcolor[rgb]{0.31,0.60,0.02}{#1}}
\newcommand{\StringTok}[1]{\textcolor[rgb]{0.31,0.60,0.02}{#1}}
\newcommand{\VariableTok}[1]{\textcolor[rgb]{0.00,0.00,0.00}{#1}}
\newcommand{\VerbatimStringTok}[1]{\textcolor[rgb]{0.31,0.60,0.02}{#1}}
\newcommand{\WarningTok}[1]{\textcolor[rgb]{0.56,0.35,0.01}{\textbf{\textit{#1}}}}
\usepackage{graphicx,grffile}
\makeatletter
\def\maxwidth{\ifdim\Gin@nat@width>\linewidth\linewidth\else\Gin@nat@width\fi}
\def\maxheight{\ifdim\Gin@nat@height>\textheight\textheight\else\Gin@nat@height\fi}
\makeatother
% Scale images if necessary, so that they will not overflow the page
% margins by default, and it is still possible to overwrite the defaults
% using explicit options in \includegraphics[width, height, ...]{}
\setkeys{Gin}{width=\maxwidth,height=\maxheight,keepaspectratio}
\setlength{\emergencystretch}{3em}  % prevent overfull lines
\providecommand{\tightlist}{%
  \setlength{\itemsep}{0pt}\setlength{\parskip}{0pt}}
\setcounter{secnumdepth}{0}
% Redefines (sub)paragraphs to behave more like sections
\ifx\paragraph\undefined\else
\let\oldparagraph\paragraph
\renewcommand{\paragraph}[1]{\oldparagraph{#1}\mbox{}}
\fi
\ifx\subparagraph\undefined\else
\let\oldsubparagraph\subparagraph
\renewcommand{\subparagraph}[1]{\oldsubparagraph{#1}\mbox{}}
\fi

% set default figure placement to htbp
\makeatletter
\def\fps@figure{htbp}
\makeatother

\usepackage{enumerate}
\usepackage{hyperref}
\usepackage{etoolbox}
\makeatletter
\providecommand{\subtitle}[1]{% add subtitle to \maketitle
  \apptocmd{\@title}{\par {\large #1 \par}}{}{}
}
\makeatother

\title{Trabalho 2 - SME0806 - Estatística Computacional}
\providecommand{\subtitle}[1]{}
\subtitle{Universidade de São Paulo}
\author{Diego G. de Paulo (10857040) \and Bruno H.da S Justino (11031621) \and Douglas S. Souza (10733820) \and Caio H. M. Schiavo (11810602) \and Vitor Gratiere Torres (10284952)}
\date{18/06/2020}

\begin{document}
\maketitle

\newpage
\tableofcontents
\newpage

\hypertarget{introduuxe7uxe3o}{%
\section{Introdução}\label{introduuxe7uxe3o}}

\hypertarget{exercuxedcio-1}{%
\section{Exercício 1}\label{exercuxedcio-1}}

\hypertarget{motivauxe7uxe3o}{%
\subsection{Motivação}\label{motivauxe7uxe3o}}

~

Neste exercício a cargo dos alunos que realizam este trabalho, realizar
estimativas pontuais e intervalares do coeficiente Gini, um indicador de
desigualdade em relação ao PIB per Capita. Este coeficiente é definido
por:

\[G = \frac{\sum_{i=1}^n\sum_{i=1}^n |x_i-x_j|}{2n^2\mu}\]

\hypertarget{metodologia}{%
\subsection{Metodologia}\label{metodologia}}

~

Para solução, será utilizado método bootstrap para amostrar valores
\(x_1^*, ..., x_n^*\) provenientes, com reposição dos valores observados
da variável Pib per Capita. Após a obtenção desses valores será aplicada
a função descrita acima para o Coeficiente de Gini a fim de obter uma
estimação pontual e um intervalo de confiança bootstrap (neste caso de
95\% de confiança).

\hypertarget{resoluuxe7uxe3o}{%
\subsection{Resolução}\label{resoluuxe7uxe3o}}

~

Para obter a estimação pontual será utilizado o seguinte resultado
(baseado numa aproximação em simulações de Monte Carlo com B amostras
bootstrap):

\[\frac{1}{B}\sum_{b=1}^B g(x_{b,1}^*, ..., x_{n,1}^*)\] Já na obtenção
das estiamativas intervalares serão selecionados os quantis 2,5 e 97,5\%
do vetor de resultados gerados para o coeficiente.

\newpage

\begin{Shaded}
\begin{Highlighting}[]
\NormalTok{gini <-}\StringTok{ }\ControlFlowTok{function}\NormalTok{(x, y) \{}
\NormalTok{  z <-}\StringTok{ }\KeywordTok{abs}\NormalTok{(x }\OperatorTok{-}\StringTok{ }\NormalTok{y)}
\NormalTok{\}}

\NormalTok{ex_}\DecValTok{1}\NormalTok{ <-}\StringTok{ }\ControlFlowTok{function}\NormalTok{(df, r) \{}

\NormalTok{  g <-}\StringTok{ }\KeywordTok{c}\NormalTok{()}
\NormalTok{  n <-}\StringTok{ }\KeywordTok{nrow}\NormalTok{(df)}

  \ControlFlowTok{for}\NormalTok{ (i }\ControlFlowTok{in} \DecValTok{1}\OperatorTok{:}\NormalTok{r) \{}
\NormalTok{    am <-}\StringTok{ }\KeywordTok{sample}\NormalTok{(}\KeywordTok{as.matrix}\NormalTok{(df[, }\DecValTok{9}\NormalTok{]), n, }\DataTypeTok{replace =}\NormalTok{ T)}
\NormalTok{    matriz <-}\StringTok{ }\KeywordTok{outer}\NormalTok{(am, am, gini)}
\NormalTok{    mu <-}\StringTok{ }\KeywordTok{mean}\NormalTok{(am)}
\NormalTok{    g[i] <-}\StringTok{ }\KeywordTok{sum}\NormalTok{(matriz) }\OperatorTok{/}\StringTok{ }\NormalTok{(}\DecValTok{2} \OperatorTok{*}\StringTok{ }\NormalTok{(n }\OperatorTok{^}\StringTok{ }\DecValTok{2}\NormalTok{) }\OperatorTok{*}\StringTok{ }\NormalTok{mu)}
\NormalTok{  \}}

  \KeywordTok{cat}\NormalTok{(}\StringTok{"Estimativa pontual para o Coeficiente Gini = "}\NormalTok{, }\KeywordTok{round}\NormalTok{(}\KeywordTok{mean}\NormalTok{(g), }\DecValTok{4}\NormalTok{),}
      \StringTok{"com"}\NormalTok{, r, }\StringTok{"repetições"}\NormalTok{, }\StringTok{"}\CharTok{\textbackslash{}n}\StringTok{"}\NormalTok{)}

  \KeywordTok{cat}\NormalTok{(}\StringTok{"Intervalo de 95% para o Coeficiente Gini:"}\NormalTok{, }\StringTok{"["}\NormalTok{,}
      \KeywordTok{round}\NormalTok{(}\KeywordTok{quantile}\NormalTok{(g, }\FloatTok{.025}\NormalTok{), }\DecValTok{4}\NormalTok{), }\StringTok{";"}\NormalTok{, }\KeywordTok{round}\NormalTok{(}\KeywordTok{quantile}\NormalTok{(g, }\FloatTok{.975}\NormalTok{), }\DecValTok{4}\NormalTok{),}
      \StringTok{"]"}\NormalTok{, }\StringTok{"com"}\NormalTok{, r, }\StringTok{"}\CharTok{\textbackslash{}n}\StringTok{repetições"}\NormalTok{, }\StringTok{"}\CharTok{\textbackslash{}n}\StringTok{"}\NormalTok{)}
\NormalTok{\}}

\KeywordTok{ex_1}\NormalTok{(}\DataTypeTok{df =}\NormalTok{ df_fim, }\DataTypeTok{r =} \DecValTok{500}\NormalTok{)}
\end{Highlighting}
\end{Shaded}

\begin{verbatim}
## Estimativa pontual para o Coeficiente Gini =  0,3254 com 500 repetições
## Intervalo de 95% para o Coeficiente Gini: [ 0,2967 ; 0,3566 ] com 500
## repetições
\end{verbatim}

\begin{Shaded}
\begin{Highlighting}[]
\KeywordTok{ex_1}\NormalTok{(}\DataTypeTok{df =}\NormalTok{ df_fim, }\DataTypeTok{r =} \DecValTok{1000}\NormalTok{)}
\end{Highlighting}
\end{Shaded}

\begin{verbatim}
## Estimativa pontual para o Coeficiente Gini =  0,3259 com 1000 repetições
## Intervalo de 95% para o Coeficiente Gini: [ 0,2948 ; 0,3578 ] com 1000
## repetições
\end{verbatim}

\begin{Shaded}
\begin{Highlighting}[]
\KeywordTok{ex_1}\NormalTok{(}\DataTypeTok{df =}\NormalTok{ df_fim, }\DataTypeTok{r =} \DecValTok{3000}\NormalTok{)}
\end{Highlighting}
\end{Shaded}

\begin{verbatim}
## Estimativa pontual para o Coeficiente Gini =  0,3257 com 3000 repetições
## Intervalo de 95% para o Coeficiente Gini: [ 0,2942 ; 0,3584 ] com 3000
## repetições
\end{verbatim}

Como foi notado, o coeficiente teve um resultado apresentado próximo a
0,3, apresentando um índice de desigualdade relativamente baixo entre as
cidades do estado em relação ao PIB per Capita.

\newpage

\hypertarget{exercuxedcio-2}{%
\section{Exercício 2}\label{exercuxedcio-2}}

\hypertarget{motivauxe7uxe3o-1}{%
\subsection{Motivação}\label{motivauxe7uxe3o-1}}

~

Para o exercício os resultados procurados são, também, uma estimação
pontual e um intervalo de confiança para a associação entre as duas
variáveis, que se enquadram em valor agregado, mais associadas entre si.

\hypertarget{metodologia-1}{%
\subsection{Metodologia}\label{metodologia-1}}

~

Para a avaliação desta associação foi selecionado o Coeficiente de
Correlação de Spearman, apresentado em forma de mapa de calor conforme
segue.

\hypertarget{resoluuxe7uxe3o-1}{%
\subsection{Resolução}\label{resoluuxe7uxe3o-1}}

~

\begin{Shaded}
\begin{Highlighting}[]
\NormalTok{corr_mat <-}\StringTok{ }\KeywordTok{cor}\NormalTok{(df_fim[, }\KeywordTok{c}\NormalTok{(}\DecValTok{2}\OperatorTok{:}\DecValTok{6}\NormalTok{)], }\DataTypeTok{method =} \StringTok{"s"}\NormalTok{)}
\KeywordTok{corPlot}\NormalTok{(corr_mat, }\DataTypeTok{cex =} \FloatTok{1.2}\NormalTok{)}
\KeywordTok{title}\NormalTok{(}\StringTok{"Mapa de Calor dos Coeficientes Correlações de Spearman"}\NormalTok{)}
\end{Highlighting}
\end{Shaded}

\includegraphics{trab2_files/figure-latex/unnamed-chunk-3-1.pdf}

Pelo o resultado obtido, as variáveis selecionadas foram Total
(exclusive Administração Pública) e Total Geral. Para a obtenção dos
resultados de interesse deste exercício, um método semelhante ao método
do exercício anterior foi utilizado com a diferença de que neste caso as
amostras das váriaves selecionadas foram amostradas de forma pareada
para a obtenção da medida de associação (vale lembrar que o Coeficiente
de Correlação de Spearman leva em conta os ranks das observações para a
obtenção do resultado).

\begin{Shaded}
\begin{Highlighting}[]
\NormalTok{ex_}\DecValTok{2}\NormalTok{ <-}\StringTok{ }\ControlFlowTok{function}\NormalTok{(df, r) \{}

\NormalTok{  g <-}\StringTok{ }\KeywordTok{c}\NormalTok{()}
\NormalTok{  n <-}\StringTok{ }\KeywordTok{nrow}\NormalTok{(df)}

  \ControlFlowTok{for}\NormalTok{ (i }\ControlFlowTok{in} \DecValTok{1}\OperatorTok{:}\NormalTok{r) \{}
\NormalTok{    index <-}\StringTok{ }\KeywordTok{sample}\NormalTok{(}\KeywordTok{c}\NormalTok{(}\DecValTok{1}\OperatorTok{:}\NormalTok{n), n, }\DataTypeTok{replace =}\NormalTok{ T)}
\NormalTok{    am1 <-}\StringTok{ }\NormalTok{df}\OperatorTok{$}\NormalTok{TotalSem[index]}
\NormalTok{    am2 <-}\StringTok{ }\NormalTok{df}\OperatorTok{$}\NormalTok{Total[index]}
\NormalTok{    g[i] <-}\StringTok{ }\KeywordTok{cor}\NormalTok{(}\DataTypeTok{x =}\NormalTok{ am1, }\DataTypeTok{y =}\NormalTok{ am2, }\DataTypeTok{method =} \StringTok{"s"}\NormalTok{)}
\NormalTok{  \}}

  \KeywordTok{cat}\NormalTok{(}\StringTok{"Estimativa pontual para o Coeficiente de Correlação de Spearman = "}\NormalTok{,}
      \KeywordTok{round}\NormalTok{(}\KeywordTok{mean}\NormalTok{(g), }\DecValTok{4}\NormalTok{), }\StringTok{"com}\CharTok{\textbackslash{}n}\StringTok{"}\NormalTok{, r, }\StringTok{"repetições"}\NormalTok{, }\StringTok{"}\CharTok{\textbackslash{}n}\StringTok{"}\NormalTok{)}

  \KeywordTok{cat}\NormalTok{(}\StringTok{"Intervalo de 95% para o Coeficiente de Correlação de Spearman:"}\NormalTok{, }\StringTok{"}\CharTok{\textbackslash{}n}\StringTok{["}\NormalTok{,}
      \KeywordTok{round}\NormalTok{(}\KeywordTok{quantile}\NormalTok{(g, }\FloatTok{.025}\NormalTok{), }\DecValTok{4}\NormalTok{), }\StringTok{";"}\NormalTok{, }\KeywordTok{round}\NormalTok{(}\KeywordTok{quantile}\NormalTok{(g, }\FloatTok{.975}\NormalTok{), }\DecValTok{4}\NormalTok{),}
      \StringTok{"]"}\NormalTok{, }\StringTok{"com"}\NormalTok{, r, }\StringTok{"repetições"}\NormalTok{, }\StringTok{"}\CharTok{\textbackslash{}n}\StringTok{"}\NormalTok{)}
\NormalTok{\}}

\KeywordTok{ex_2}\NormalTok{(}\DataTypeTok{df =}\NormalTok{ df_fim, }\DataTypeTok{r =} \DecValTok{500}\NormalTok{)}
\end{Highlighting}
\end{Shaded}

\begin{verbatim}
## Estimativa pontual para o Coeficiente de Correlação de Spearman =  0,9873 com
##  500 repetições
## Intervalo de 95% para o Coeficiente de Correlação de Spearman:
## [ 0,9837 ; 0,9902 ] com 500 repetições
\end{verbatim}

\begin{Shaded}
\begin{Highlighting}[]
\KeywordTok{ex_2}\NormalTok{(}\DataTypeTok{df =}\NormalTok{ df_fim, }\DataTypeTok{r =} \DecValTok{1000}\NormalTok{)}
\end{Highlighting}
\end{Shaded}

\begin{verbatim}
## Estimativa pontual para o Coeficiente de Correlação de Spearman =  0,9873 com
##  1000 repetições
## Intervalo de 95% para o Coeficiente de Correlação de Spearman:
## [ 0,9836 ; 0,9901 ] com 1000 repetições
\end{verbatim}

\begin{Shaded}
\begin{Highlighting}[]
\KeywordTok{ex_2}\NormalTok{(}\DataTypeTok{df =}\NormalTok{ df_fim, }\DataTypeTok{r =} \DecValTok{3000}\NormalTok{)}
\end{Highlighting}
\end{Shaded}

\begin{verbatim}
## Estimativa pontual para o Coeficiente de Correlação de Spearman =  0,9873 com
##  3000 repetições
## Intervalo de 95% para o Coeficiente de Correlação de Spearman:
## [ 0,9837 ; 0,9901 ] com 3000 repetições
\end{verbatim}

Como é observado, de acordo com os resultados obtidos, a correlação está
bem próxima de 1, que vai bem de encontro ao resultado visualizado no
mapa de calor, ou seja, já era um resultado esperado.

\hypertarget{conclusuxe3o}{%
\section{Conclusão}\label{conclusuxe3o}}

Após a realização deste trabalho, o grupo de alunos responsável pode
fixar o conteúdo de reamostragem com grande enfoque no método boostrap,
principalmente pela utilização de dados reais obtidos a partir de um
banco de dados obtidos disponibilizado pelo governo.

\end{document}
